\documentclass{beamer}
\usepackage[british,spanish]{babel}
\usepackage[utf8]{inputenc}
\usepackage[british,spanish]{babel}
%usepackage[utf8]{inputenx}

\usepackage{hyperref}
%\hypersetup{colorlinks=false,linkbordercolor=red,linkcolor=green,pdfborderstyle={/S/U/W 1}}

\usepackage{multirow}

\usepackage{textcomp}

\usepackage{listings}
\lstloadlanguages{Ruby}
\usepackage{cancel}

\usepackage{adjustbox}
\usepackage{lstcustom}

\usepackage{amsmath}

\usepackage{color}
\definecolor{light-gray}{gray}{0.80}
\definecolor{lstbackgroundshellcolor}{named}{light-gray}

\usepackage{tikz}
\newcommand*\circled[1]{\tikz[baseline=(char.base)]{
            \node[shape=circle,draw,inner sep=2pt] (char) {#1};}}

\usepackage[normalem]{ulem}

%\usepackage[acronym,xindy,toc]{glossaries}

\usepackage[acronym,xindy,toc]{glossaries}
\makeglossaries
%\usepackage[xindy]{imakeidx}
%\makeindex


\newcommand{\comment}[2]{#2}

\newcommand{\commandinline}[1]{\lstinline[basicstyle=\small\lstfontfamily]{#1}}
\newcommand{\outputcommand}[1]{\color{darkgreen}{#1}}

\graphicspath{ {./images/} }

\title{Amazon SimpleDB: A Cloud Database}
%\subtitle[short subtitle]{long subtitle}
\author[C. Cuenca, F. Quintana]{Carmelo Cuenca-Hernández and Francisca Quintana-Domínguez}
%\institute{Escuela Universitaria de Informática}
%\date[04/2013]{Abril - 2013}
\date{}
\titlegraphic{\includegraphics[width=0.5 \textwidth]{./images/logo_ulpgc_version_horizontal_rgb.eps}}


\pgfdeclareimage[width=2.0\baselineskip]{ulpgc-logo}{images/logosimbolo_secundario_version_vertical}
\setbeamertemplate{footline}{\raisebox{-2ex}{\pgfuseimage{ulpgc-logo}}
  \usebeamerfont{date in head/foot}\insertshortdate{}\hfill
  \usebeamertemplate{navigation symbols}\hfill
  \insertframenumber{}/\inserttotalframenumber}
\setbeamertemplate{sidebar right}{}


\usetheme{Antibes}
%\usetheme{Berlin}

%\usetheme{Warsaw}
%\usecolortheme{albatross}

\selectlanguage{british}



\begin{document}

%\includegraphics[width= 1.0 \textwidth]{logos3.eps}
\begin{frame}
	\titlepage
\end{frame}


\section*{Outline}
\begin{frame}[fragile, allowframebreaks]
  \frametitle{Outline}
  %\tableofcontents%[part=1,pausesections]
  \tableofcontents[currentsection,currentsubsection, sectionstyle=show] 
  %\tableofcontents[currentsection,sectionstyle=show,hideothersubsections]
\end{frame}

%%%%%%%%%%%%%%%%%%%%%%%%%%%%%%%%%%%%%%%%%%%%%%%%%%%%%%%%%%%%%%%%%%%%%%%%%%%%%%
%\newacronym{<label>}{<abbrv>}{<full>}
%\glsreset{<label>}
%\glsresetall
%\acrlong{<label>}
%\acrfull{<label>}
%\acrshort{<label>}
%\input{../glossary}

\newacronym{acl}{ACL}{Access Control List}
\newacronym{api}{API}{Application Programming Interface}
\newacronym{aws}{AWS}{Amazon Web Services}
\newacronym{cli}{CLI}{Command Line Interface}
\newacronym{css}{CSS}{cascading style sheets}
\newacronym{ebs}{EBS}{Elastic Block StoAWS Identity an Access Managementrage}
\newacronym{ec2}{EC2}{Amazon Elastic Compute Cloud}
\newacronym{elb}{ELB}{Elastic Load Balancing}
\newacronym{iam}{IAM}{Identity Access Management}
\newacronym{ror}{RoR}{{\href{http://rubyonrails.org/}{Ruby on Rails}}}
\newacronym{rds}{RDS}{Relational Database Service}
\newacronym{rvm}{RVM}{{\href{https://rvm.io/}{Ruby Version Manager}}}
\newacronym{s3}{S3}{Simple Storage Service}
\newacronym{sqs}{SQS}{Amazon Simple Queue Service}
%%%%%%%%%%%%%%%%%%%%%%%%%%%%%%%%%%%%%%%%%%%%%%%%%%%%%%%%%%%%%%%%%%%%%%%%%%%%%%
\section{Amazon SimpleDB}
\begin{frame}[fragile]
\frametitle{Amazon SimpleDB}
\begin{itemize}
 \item Amazon SimpleDB is a cloud-based database
 \item SimpleDB excels at storing semi-structured data where the items (rows) are similar but not necessarily identical to each other
 \item All data stored in Amazon SimpleDB is automatically indexed
 \item The Amazon SimpleDB model lacks support for joins across domains (roughly
equivalent to relational tables); instead, you can store data in non-normalized form
for more efficient access
\end{itemize}
\end{frame}
%%%%%%%%%%%%%%%%%%%%%%%%%%%%%%%%%%%%%%%%%%%%%%%%%%%%%%%%%%%%%%%%%%%%%%%%%%%%%%
\begin{frame}[fragile]
\frametitle{Amazon SimpleDB Concepts}
\begin{itemize}
 \item An Amazon SimpleDB \emph{domain} is roughly analogous to a table in a relational database.  Each domain can store up to \texttt{10GB} of data
 \item SimpleDB excels at storing semi-structured data where the items (rows) are similar but not necessarily identical to each other
 \item Each item in a SimpleDB domain has a name (unique to the domain) and up to \texttt{256} attributes (name-value pairs).
       Item names, attribute names, and attribute values can each be up to \texttt{1024} bytes long
 \item Item attributes can be multivalued 
 \item All attribute values are treated as strings (\alert{numerical values?})
\end{itemize}
\end{frame}

%%%%%%%%%%%%%%%%%%%%%%%%%%%%%%%%%%%%%%%%%%%%%%%%%%%%%%%%%%%%%%%%%%%%%%%%%%%%%%
%%%%%%%%%%%%%%%%%%%%%%%%%%%%%%%%%%%%%%%%%%%%%%%%%%%%%%%%%%%%%%%%%%%%%%%%%%%%%%
\begin{frame}[fragile]
\frametitle{\texttt{Select}}
\begin{tabular}{|l|l|l|l|l|l|l|} \hline
item\_name & first\_name & last\_name & age & sex & middle & state \\ \hline
Rec1 & Tom & Basic & 12 & M & & \\ \hline
Rec2 & Nancy & Hacker & 15 & F & & MD \\ \hline 
Rec3 & Joan & Hughes & 44 & F & Fraft & NY, NJ\\ \hline 
\end{tabular} 

\begin{itemize}
 \item \texttt{select * from People where FirstName="Tom"} (Rec1)
 \item \texttt{select FirstName,Middle,LastName from People where LastName >= "H"} (Rec2 and Rec3)
 \end{itemize}
\end{frame}
%%%%%%%%%%%%%%%%%%%%%%%%%%%%%%%%%%%%%%%%%%%%%%%%%%%%%%%%%%%%%%%%%%%%%%%%%%%%%%
%%%%%%%%%%%%%%%%%%%%%%%%%%%%%%%%%%%%%%%%%%%%%%%%%%%%%%%%%%%%%%%%%%%%%%%%%%%%%%
\section{Amazon SimpleDB Pricing}
\begin{frame}[fragile]
\frametitle{Amazon SimpleDB Pricing}

\begin{itemize}
 \item Data transfer. Data transferred into SimpleDB is charged at a rate of \$0.10 per gigabyte
 \item Data storage. You pay \$0. 25 ( a quarter) per gigabyte per month to store data in Simple DB
 \item Machine utilization. You pay \$0.14 per hour for the machine time used to process each SimpleDB
request
 \end{itemize}
\end{frame}

%%%%%%%%%%%%%%%%%%%%%%%%%%%%%%%%%%%%%%%%%%%%%%%%%%%%%%%%%%%%%%%%%%%%%%%%%%%%%%
\section{Amazon SimpleDB Programming Model}
\begin{frame}[fragile]
\frametitle{Amazon SimpleDB Programming Model}
\begin{itemize}
 \item The Amazon SimpleDB programming model consists of just nine calls:
 \begin{itemize}
 \item At the domain level: \texttt{CreateDomain}, \texttt{DeleteDomain}, \texttt{ListDomains}, \texttt{DomainMetadata}
 \item At the item level
 \begin{itemize}
 \item \texttt{PutAttributes} creates new items and adds or replaces (your choice) additional attributes to existing items
 \item \texttt{BatchPutAttributes} is an extended version of this call that handles multiple items at once
 \item \texttt{DeleteAttributes} removes attributes from an item
 \item \texttt{GetAttributes} retrieve specified attributes
 \item \texttt{Select} issues a SQL query
 \end{itemize}
\end{itemize}

\end{itemize}
\end{frame}
%%%%%%%%%%%%%%%%%%%%%%%%%%%%%%%%%%%%%%%%%%%%%%%%%%%%%%%%%%%%%%%%%%%%%%%%%%%%%%
%%%%%%%%%%%%%%%%%%%%%%%%%%%%%%%%%%%%%%%%%%%%%%%%%%%%%%%%%%%%%%%%%%%%%%%%%%%%%%
\begin{frame}[fragile, allowframebreaks]
\frametitle{Create a Domain}
\begin{itemize}
\item Here’ s a script that creates all the SimpleDB domains needed for the examples
\lstset{language=Ruby, style=eclipse}
\begin{lstlisting}[escapechar=!]
#
# aws_app/simpledb/book.inc.rb
#
# This file is required by some *.rb files
# It defines default constant value

BOOK_FILE_DOMAIN = 'files'
BOOK_FEED_DOMAIN = 'feeds'
BOOK_FEED_ITEM_DOMAIN = 'feed_items'
\end{lstlisting}
\lstset{language=Ruby, style=eclipse}
\begin{lstlisting}[escapechar=!]
#
# aws_app/simpledb/create_domain.rb
#
# Usage: ruby simpledb/create_domain.rb
#

require File.expand_path("#{File.dirname(__FILE__)}/../config")
require File.expand_path("#{File.dirname(__FILE__)}/book.inc")

#get an instance of the SDB interface using the default configuration
sdb = AWS::SimpleDB.new(region: 'us-east-1')

[BOOK_FILE_DOMAIN, BOOK_FEED_DOMAIN, BOOK_FEED_ITEM_DOMAIN].each do |domain_name|
  begin
    domain = sdb.domains.create(domain_name)
      puts "Domain #{domain.name} created"
  rescue Exception => e
    puts e.message
  end
end
\end{lstlisting}

\item Modify the above script to accept new domain names on the command line (Exercise)
\item Install SDBTool plugin in Firefox from http://code.google.com/p/sdbtool/ and use it (Exercise)
\end{itemize}

\end{frame}
%%%%%%%%%%%%%%%%%%%%%%%%%%%%%%%%%%%%%%%%%%%%%%%%%%%%%%%%%%%%%%%%%%%%%%%%%%%%%%
\begin{frame}[fragile]
\frametitle{List Domains}

\lstset{language=Ruby, style=eclipse}
\begin{lstlisting}[escapechar=!]
#
# aws_app/simpledb/list_domains.rb
#
# Usage: ruby simpledb/lists_domains.rb
#

require File.expand_path(File.dirname(__FILE__) + '/../config')
require File.expand_path(File.dirname(__FILE__) + '/book.inc')

# get an instance of the SDB interface using the default configuration
sdb = AWS::SimpleDB.new region: 'us-east-1'

sdb.domains.each do |domain|
  puts domain.name
end
\end{lstlisting}

\end{frame}
%%%%%%%%%%%%%%%%%%%%%%%%%%%%%%%%%%%%%%%%%%%%%%%%%%%%%%%%%%%%%%%%%%%%%%%%%%%%%%
\begin{frame}[fragile, allowframebreaks]
\frametitle{Storing Data}
\begin{itemize}
\item The next step is to store some data in a domain. This is done using the
\texttt{put\_attributes} method. This method can be used to create new items or to add
additional attributes to existing items

\lstset{language=Ruby, style=eclipse}
\begin{lstlisting}[escapechar=!]
#
# aws_app/simpledb/insert_items.rb
#
# Usage: ruby simpledb/insert_items.rb
#

require File.expand_path(File.dirname(__FILE__) + '/../config')
require File.expand_path(File.dirname(__FILE__) + '/book.inc')

# get an instance of the SDB interface using the default configuration
sdb = AWS::SimpleDB.new :region => 'us-east-1'

Dir.entries('.').select{ |file| file.match(/^[a-zA-Z0-9_-]*\.rb$/)}.each do |filename|
  File.open filename do |f|
    data = f.read
    size = f.size
    md5 = Digest::MD5.hexdigest(data)

    item = sdb.domains[BOOK_FILE_DOMAIN].items[filename]
    item.attributes.put(
      add: { name: filename }, # Append values to this attribute
      replace: { md5: md5, size: "%08d" % size}  # Set all values for this attribute, replacing current values
    )
  end
end
\end{lstlisting}
\end{itemize}
\end{frame}
%%%%%%%%%%%%%%%%%%%%%%%%%%%%%%%%%%%%%%%%%%%%%%%%%%%%%%%%%%%%%%%%%%%%%%%%%%%%%%
%%%%%%%%%%%%%%%%%%%%%%%%%%%%%%%%%%%%%%%%%%%%%%%%%%%%%%%%%%%%%%%%%%%%%%%%%%%%%%
\begin{frame}[fragile, allowframebreaks]
\frametitle{Storing Multiple Items Efficiently}
\begin{itemize}
\item The next step is to store some data in a domain. This is done using the
\texttt{put\_attributes} method. This method can be used to create new items or to add
additional attributes to existing items

\lstset{language=Ruby, style=eclipse}
\begin{lstlisting}[escapechar=!]

\end{lstlisting}
\end{itemize}
\end{frame}
%%%%%%%%%%%%%%%%%%%%%%%%%%%%%%%%%%%%%%%%%%%%%%%%%%%%%%%%%%%%%%%%%%%%%%%%%%%%%%
\end{document}

