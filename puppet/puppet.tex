\documentclass{beamer}
\usepackage[british,spanish]{babel}
\usepackage[utf8]{inputenc}
\usepackage{hyperref}
\usepackage{multirow}



\usepackage{listings}

\usepackage{adjustbox}
\usepackage{lstcustom}

\usepackage{color}
\definecolor{light-gray}{gray}{0.80}
\definecolor{lstbackgroundshellcolor}{named}{light-gray}

\usepackage{tikz}
\newcommand*\circled[1]{\tikz[baseline=(char.base)]{
            \node[shape=circle,draw,inner sep=2pt] (char) {#1};}}

\usepackage[normalem]{ulem}

\newcommand{\outputcommand}[1]{\color{darkgreen}{#1}}
\usepackage[acronym,xindy,toc]{glossaries}
\makeglossaries
%\usepackage[xindy]{imakeidx}
%\makeindex

\newcommand{\comment}[2]{#2}

\graphicspath{ {./images/} }

\title[Puppet]{Building reliable, scalable, secure, and high-perfomance Systems to fully utilize the power of the cloud computing}
%\subtitle[short subtitle]{long subtitle}
\author[C. Cuenca, F. Quintana]{Carmelo Cuenca-Hernández and Francisca Quintana-Domínguez}
%\institute{Escuela Universitaria de Informática}
%\date[04/2013]{Abril - 2013}
\date{}
\titlegraphic{\includegraphics[width=0.5 \textwidth]{images/puppet-labs-logo.png}}



\pgfdeclareimage[width=2.0\baselineskip]{ulpgc-logo}{images/logosimbolo_secundario_version_vertical}
\setbeamertemplate{footline}{\raisebox{-2ex}{\pgfuseimage{ulpgc-logo}}
  \usebeamerfont{date in head/foot}\insertshortdate{}\hfill
  \usebeamertemplate{navigation symbols}\hfill
  \insertframenumber{}/\inserttotalframenumber}
\setbeamertemplate{sidebar right}{}


\usetheme{Antibes}
%\usetheme{Berlin}

%\usetheme{Warsaw}
%\usecolortheme{albatross}

\begin{document}

\begin{frame}
	\titlepage
\end{frame}


\section*{Outline}
\begin{frame}
  \frametitle{Outline}
  %\tableofcontents%[part=1,pausesections]
  %\tableofcontents[currentsection,currentsubsection, sectionstyle=show] 
  \tableofcontents[currentsection,sectionstyle=show,hideothersubsections]
\end{frame}


\selectlanguage{british}

%%%%%%%%%%%%%%%%%%%%%%%%%%%%%%%%%%%%%%%%%%%%%%%%%%%%%%%%%%%%%%%%%%%%%%%%%%%%%%
%\newacronym{<label>}{<abbrv>}{<full>}
%\glsreset{<label>}
%\glsresetall
%\acrlong{<label>}
%\acrfull{<label>}
%\acrshort{<label>}
\newacronym{aws}{AWS}{Amazon Web Services}
\newacronym{ebs}{EBS}{Elastic Block Storage}
\newacronym{ec2}{EC2}{Amazon Elastic Compute Cloud}
\newacronym{ecu}{ECU}{Elastic Compute Unit}
\newacronym{elb}{ELB}{Elastic Load Balancing}
\newacronym{rds}{RDS}{Relational Database Service}
\newacronym{s3}{S3}{Simple Storage Service}
\newacronym{sqs}{SQS}{Amazon Simple Queue Service}
%%%%%%%%%%%%%%%%%%%%%%%%%%%%%%%%%%%%%%%%%%%%%%%%%%%%%%%%%%%%%%%%%%%%%%%%%%%%%%
%%%%%%%%%%%%%%%%%%%%%%%%%%%%%%%%%%%%%%%%%%%%%%%%%%%%%%%%%%%%%%%%%%%%%%%%%%%%%%
\begin{frame}[fragile, allowframebreaks]
\frametitle{Puppet}
\begin{itemize}
\item \url{http://puppetlabs.com/}
\item Puppet lets you describe configuration in terms of resources—what things should exist—and
their attributes. You don't have to get into the details of how resources are created and
configured on different platforms. Puppet just takes care of it
\item Here are some of the kinds of resources you can describe in Puppet
\begin{itemize}
\item Packages, Files, Services, Users, Groups, YUM repos, Nagios configuration, Log messages, /etc/hosts entries, Network interfaces, SSH keys,
SELinux settings, Kerberos configuration, ZFS attributes, E-mail aliases, Mailing lists, Mounted filesystems,
Scheduled jobs, VLANs, Solaris zones
\end{itemize}
\item The Puppet advantage
\begin{itemize}
\item You can write a manifest once and apply it to many machines, avoiding
duplicated work
\item You can keep all your servers in sync with each other, and with the manifest
\item The Puppet manifest also acts as live documentation, which is guaranteed to
be up to date
\item Puppet copes with differences between operating systems, platforms, command
syntaxes, and so on
\end{itemize}
\end{itemize}
\end{frame}
%%%%%%%%%%%%%%%%%%%%%%%%%%%%%%%%%%%%%%%%%%%%%%%%%%%%%%%%%%%%%%%%%%%%%%%%%%%%%%
\section{Up and Running}
%%%%%%%%%%%%%%%%%%%%%%%%%%%%%%%%%%%%%%%%%%%%%%%%%%%%%%%%%%%%%%%%%%%%%%%%%%%%%%
\begin{frame}[fragile, allowframebreaks]
\frametitle{Up and Running}
\url{http://docs.puppetlabs.com/guides/puppetlabs_package_repositories.html#for-red-hat-enterprise-linux-and-derivatives}

\begin{itemize}
\item To enable the repository, run the command below that corresponds to your OS version and architecture:
\begin{itemize}
 \item Enterprise Linux 6 i386

\lstset{language=shell}
\begin{lstlisting}[escapechar=!]
$ sudo rpm -ivh https://yum.puppetlabs.com/el/6/products/i386/puppetlabs-release-6-7.noarch.rpm
\end{lstlisting}

\item Enterprise Linux 6 x86\_64
\lstset{language=shell}
\begin{lstlisting}[escapechar=!]
$ sudo rpm -ivh https://yum.puppetlabs.com/el/6/products/x86_64/puppetlabs-release-6-7.noarch.rpm
\end{lstlisting}
\end{itemize}
\lstset{language=shell}
\begin{lstlisting}[escapechar=!]
$ sudo yum install puppet facter -y
$ puppet --version
!\outputcommand{3.4.2}!
\end{lstlisting}
\end{itemize}
\end{frame}
%%%%%%%%%%%%%%%%%%%%%%%%%%%%%%%%%%%%%%%%%%%%%%%%%%%%%%%%%%%%%%%%%%%%%%%%%%%%%%
%%%%%%%%%%%%%%%%%%%%%%%%%%%%%%%%%%%%%%%%%%%%%%%%%%%%%%%%%%%%%%%%%%%%%%%%%%%%%%
\section{Hello World}
%%%%%%%%%%%%%%%%%%%%%%%%%%%%%%%%%%%%%%%%%%%%%%%%%%%%%%%%%%%%%%%%%%%%%%%%%%%%%%
\begin{frame}[fragile, allowframebreaks]
\frametitle{Hello World!}
\begin{columns}
\column{ 0.5 \textwidth}
\lstset{language=shell}
\begin{lstlisting}[escapechar=&]
# site.pp
file { '/tmp/hello.text':
  content => 'Hello World!',
}
\end{lstlisting}
\column{ 0.5 \textwidth}
\lstset{language=shell}
\begin{lstlisting}[escapechar=!]
RESOURCE { NAME:
  ATTRIBUTE => VALUE,
  !\dots!
}
\end{lstlisting}
\end{columns}

\lstset{language=shell}
\begin{lstlisting}[escapechar=&]
$ puppet apply site.pp
&\outputcommand{Notice: Compiled catalog for ponderosa.lan in environment production in 0.35 seconds\\
Notice: /Stage[main]/Main/File[/tmp/hello.text]/ensure: defined content as '{md5}bcdad52d22f1510b7c9389176f80f4fa'\\
Notice: Finished catalog run in 0.28 seconds}&
$ cat /tmp/hello.txt
&\outputcommand{Hello World!}&
\end{lstlisting}

\lstset{language=shell}
\begin{lstlisting}[escapechar=&]
$ puppet apply site.pp
&\outputcommand{Notice: Compiled catalog for ponderosa.lan in environment production in 0.25 seconds\\
Notice: Finished catalog run in 0.20 seconds\\
}&
$ cat /tmp/hello.txt
&\outputcommand{Hello World!}&
\end{lstlisting}

\lstset{language=shell}
\begin{lstlisting}[escapechar=&]
$ echo 'Goodbye' > /tmp/hello.txt
$ puppet apply site.pp
&\outputcommand{Notice: Compiled catalog for ponderosa.lan in environment production in 0.23 seconds\\
Notice: /Stage[main]/Main/File[/tmp/hello.text]/content: content changed '{md5}a05945549d7a0eda92c8a1755587346a' to '{md5}bcdad52d22f1510b7c9389176f80f4fa'\\
Notice: Finished catalog run in 0.25 seconds}&
$ cat /tmp/hello.txt
&\outputcommand{Hello World!}&
\end{lstlisting}

\end{frame}
%%%%%%%%%%%%%%%%%%%%%%%%%%%%%%%%%%%%%%%%%%%%%%%%%%%%%%%%%%%%%%%%%%%%%%%%%%%%%%
%%%%%%%%%%%%%%%%%%%%%%%%%%%%%%%%%%%%%%%%%%%%%%%%%%%%%%%%%%%%%%%%%%%%%%%%%%%%%%
%%%%%%%%%%%%%%%%%%%%%%%%%%%%%%%%%%%%%%%%%%%%%%%%%%%%%%%%%%%%%%%%%%%%%%%%%%%%%%
\section{Nodes}
%%%%%%%%%%%%%%%%%%%%%%%%%%%%%%%%%%%%%%%%%%%%%%%%%%%%%%%%%%%%%%%%%%%%%%%%%%%%%%
\begin{frame}[fragile]
\frametitle{Nodes}
\begin{itemize}
\item Creating a directory structure
\lstset{language=shell}
\begin{lstlisting}[escapechar=&]
$ cd
$ mkdir -p puppet/manifests
$ mv site.pp puppet/manifests
$ puppet apply manifests/site.pp
\end{lstlisting}

\item ``node'' is the Puppet term for an individual machine that has a Puppet configuration

\lstset{language=shell}
\begin{lstlisting}[escapechar=&]
node NODENAME {
  RESOURCE
  RESOURCE
  &\dots&
}
\end{lstlisting}
\item Here ``NODENAME'' is the hostname of the relevant machine, and ``RESOURCE'' is a resource
declaration
\end{itemize}
\end{frame}
%%%%%%%%%%%%%%%%%%%%%%%%%%%%%%%%%%%%%%%%%%%%%%%%%%%%%%%%%%%%%%%%%%%%%%%%%%%%%%
%%%%%%%%%%%%%%%%%%%%%%%%%%%%%%%%%%%%%%%%%%%%%%%%%%%%%%%%%%%%%%%%%%%%%%%%%%%%%%
\begin{frame}[fragile]
\frametitle{Creating a node declaration}
\begin{itemize}
\item Create the file \texttt{manifests/nodes.pp} with the following contents
\lstset{language=shell}
\begin{lstlisting}[escapechar=&]
# manifests/nodes.pp

node 'ponderosa' {
  file { '/tmp/hello.txt':
    content => 'Hello World!',
  }
}
\end{lstlisting}
\item Change the manifests/site.pp file so it contains:
\lstset{language=shell}
\begin{lstlisting}[escapechar=&]
# manifest/site.pp

import 'nodes.pp'
\end{lstlisting}

\item Check whether everything still works:
\lstset{language=shell}
\begin{lstlisting}[escapechar=&]
$ puppet apply manifests/site.pp
\end{lstlisting}

\end{itemize}
\end{frame}
%%%%%%%%%%%%%%%%%%%%%%%%%%%%%%%%%%%%%%%%%%%%%%%%%%%%%%%%%%%%%%%%%%%%%%%%%%%%%%
%%%%%%%%%%%%%%%%%%%%%%%%%%%%%%%%%%%%%%%%%%%%%%%%%%%%%%%%%%%%%%%%%%%%%%%%%%%%%%
\section{Packages, Files and Services}

\begin{frame}[fragile]
\frametitle{Packages}
\begin{itemize}
 \item Installing Apache
\begin{itemize}
\item Edit the file \texttt{manifests/nodes.pp} with the following contents
\lstset{language=shell}
\begin{lstlisting}[escapechar=&]
# manifests/nodes.pp

node 'ponderosa' {
  package { 'httpd':
    ensure => installed, # absent
  }
}
\end{lstlisting}
where there are a type of resource: ``package'', the name of the instance: ``nginx'', and a list of attributes
\item Apply your manifest
\lstset{language=shell}
\begin{lstlisting}[escapechar=&]
$ sudo puppet apply manifests/site.pp 
\end{lstlisting}
\end{itemize}
\end{itemize}
\end{frame}
%%%%%%%%%%%%%%%%%%%%%%%%%%%%%%%%%%%%%%%%%%%%%%%%%%%%%%%%%%%%%%%%%%%%%%%%%%%%%%
\begin{frame}[fragile]
\frametitle{Modules}
\begin{itemize}
\item To make your Puppet manifests more readable and maintainable, it's a good idea to arrange
them into modules. A Puppet module is a way of grouping related resources
\item In our example, we're going to make an htppd module that will contain all Puppet code relating to httpd

\item Create the directory \texttt{modules/httpd/manifests/init.pp}
\lstset{language=shell}
\begin{lstlisting}[escapechar=&]
$ mkdir -p modules/httpd/{manifests,files,templates}
\end{lstlisting}

\item Create the file modules/httpd/manifests/init.pp with the following
contents

\lstset{language=shell}
\begin{lstlisting}[escapechar=&]
# modules/httpd/manifests/init.pp

# Manage httpd webserver
class httpd {
  package { 'httpd':
    ensure => installed, # absent
  }
}
\end{lstlisting}

\item Edit the \texttt{manifests/nodes.pp} file as follows:
\lstset{language=shell}
\begin{lstlisting}[escapechar=&]
# manifests/nodes.pp

node 'ponderosa' {
  include httpd
}
\end{lstlisting}

\item Run Puppet to make sure everything is correct. There should be no changes
\begin{lstlisting}[escapechar=&]
$ sudo puppet apply --modulepath=${HOME}/puppet/modules manifests/site.pp
\end{lstlisting}
\end{itemize}

\end{frame}
%%%%%%%%%%%%%%%%%%%%%%%%%%%%%%%%%%%%%%%%%%%%%%%%%%%%%%%%%%%%%%%%%%%%%%%%%%%%%%
%%%%%%%%%%%%%%%%%%%%%%%%%%%%%%%%%%%%%%%%%%%%%%%%%%%%%%%%%%%%%%%%%%%%%%%%%%%%%%
\begin{frame}[fragile,allowframebreaks]
\frametitle{Services}
\begin{itemize}
\item So we're using a module to manage httpd on the server. That's great, but so far we've only
installed the httpd package. In order to run the web server, we would need to start and stop
it manually using the command line. Fortunately, we can automate this with Puppet as well

\item Edit the file \texttt{modules/httpd/manifests/init.pp} with the following
contents

\lstset{language=shell}
\begin{lstlisting}[escapechar=&]
# modules/httpd/manifests/init.pp

# Manage httpd webserver
class httpd {
  package { 'nginx':
    ensure => absent,
  }

  package { 'httpd':
    ensure => installed, # absent
    require => Package['nginx'],
  }

  service { 'httpd':
    enable => true, # Setting enable => true will configure the service to start at boot time
    ensure => running,
    require => Package['httpd'],
  }
}
\end{lstlisting}

\item Run Puppet to make sure everything is correct. There should be no changes
\lstset{language=shell}
\begin{lstlisting}[escapechar=&]
$ sudo puppet apply --modulepath=${HOME}/puppet/modules manifests/site.pp
\end{lstlisting}
\end{itemize}

\end{frame}
%%%%%%%%%%%%%%%%%%%%%%%%%%%%%%%%%%%%%%%%%%%%%%%%%%%%%%%%%%%%%%%%%%%%%%%%%%%%%%
%%%%%%%%%%%%%%%%%%%%%%%%%%%%%%%%%%%%%%%%%%%%%%%%%%%%%%%%%%%%%%%%%%%%%%%%%%%%%%
\begin{frame}[fragile, allowframebreaks]
\frametitle{More about Services}
\begin{itemize}
\item Service status options
\begin{itemize}
\item Puppet will use the service's own control script to determine whether the service is running,
by calling service SERVICENAME status (at least on UNIX-like systems)
\item If a service's control script doesn't support a status command, you can set hasstatus \texttt{=>}
false for the service resource. In this case, Puppet will look in the system process table to
see if the service is running
\item If you need Puppet to search the process table for something other than the service's name,
you can specify what to search for using the pattern attribute.
\item If searching the process table won't work, you can provide a command for Puppet to use to
determine the service's status, using the status attribute
\end{itemize}
\item Service control commands
\begin{itemize}
\item If you want to restart a service some other way than just stopping and starting the service,
you can give Puppet the command you want to use via the restart attribute.
\item You can also specify custom service start and stop commands using the start and
stop attributes
\end{itemize}
\end{itemize}

\end{frame}
%%%%%%%%%%%%%%%%%%%%%%%%%%%%%%%%%%%%%%%%%%%%%%%%%%%%%%%%%%%%%%%%%%%%%%%%%%%%%%
%%%%%%%%%%%%%%%%%%%%%%%%%%%%%%%%%%%%%%%%%%%%%%%%%%%%%%%%%%%%%%%%%%%%%%%%%%%%%%
\begin{frame}[fragile, allowframebreaks]
\frametitle{Files}
\begin{itemize}
\item We have to have Puppet install a config file on the server to define \texttt{/etc/httpd/conf/httpd.conf}
\lstset{language=shell}
\begin{lstlisting}[escapechar=&]
$ mkdir -p modules/httpd/files/etc/httpd/conf/
$ cp /etc/httpd/conf/httpd.conf modules/httpd/files/etc/httpd/conf
$ sudo touch /etc/httpd/conf/httpd.conf
\end{lstlisting}

\lstset{language=shell}
\begin{lstlisting}[escapechar=&]
# modules/httpd/manifests/init.pp

# Manage httpd webserver
class httpd {
  package { 'nginx':
    ensure => absent,
  }

  package { 'httpd':
    ensure => installed, # absent
    require => Package['nginx'],
  }

  service { 'httpd':
    enable => true, # Setting enable => true will configure the service to start at boot time
    ensure => running,
    require => Package['httpd'],
  }

  file { "/etc/httpd/conf/httpd.conf":
    owner => "root",
    group => "root",
    mode => 0640,
    # "puppet://$puppetserver/modules/httpd/etc/httpd/httpd.conf",
    source => 'puppet:///modules/httpd/etc/httpd/conf/httpd.conf',
    require => Package['httpd'],
    notify => Service['httpd'],
  }
}
\end{lstlisting}

\begin{lstlisting}[escapechar=&]
$ sudo puppet apply --modulepath=${HOME}/puppet/modules manifests/site.p
&\outputcommand{Notice: Compiled catalog for ponderosa.lan in environment production in 0.80 seconds\\
Notice: /Stage[main]/Httpd/File[/etc/httpd/conf/httpd.conf]/mode: mode changed '0644' to '0640'\\
Notice: /Stage[main]/Httpd/Service[httpd]/ensure: ensure changed 'stopped' to 'running'\\
Notice: Finished catalog run in 2.19 seconds}&
\end{lstlisting}

\item  We have to have Puppet install a greeting file on the server to define \texttt{/var/www/html/index.html}

\lstset{language=shell}
\begin{lstlisting}[escapechar=&]
$ mkdir -p modules/httpd/files/var/www/html/
$ cp echo 'Hello World! in /var/www/html/index.htlm' > modules/httpd/files/var/www/html/index.html
\end{lstlisting}

\lstset{language=shell}
\begin{lstlisting}[escapechar=&]
# modules/httpd/manifests/init.pp
&\vdots&
  file { "/var/www/html/index.html":
    owner => "root",
    group => "root",
    mode => 0644,
    source => 'puppet:///modules/httpd/var/www/html/index.html',
    require => Package['httpd'],
  }
}
\end{lstlisting}

\item Aplly the manifests and curl the localhost

\lstset{language=shell}
\begin{lstlisting}[escapechar=&]
$ sudo puppet apply --modulepath=${HOME}/puppet/modules manifests/site.p
$ curl http://localhost/index.html
\end{lstlisting}


\end{itemize}
\end{frame}
%%%%%%%%%%%%%%%%%%%%%%%%%%%%%%%%%%%%%%%%%%%%%%%%%%%%%%%%%%%%%%%%%%%%%%%%%%%%%%
\begin{frame}[fragile]
\frametitle{The package–file–service pattern}
\begin{itemize}
\item The pattern you've just learnt is a very useful one. It'll cover most services that you
need to automate
\lstset{language=shell}
\begin{lstlisting}[escapechar=&]
class THE_STUFF {
  package { THE_STUFF:
    ensure => installed,
  }
  service { THE_STUFF:
    ensure => running,
   require => Package[THE_STUFF],
  }
  file { '/etc/THE_STUFF.conf':
    source => 'puppet:///modules/THE_STUFF/THE_STUFF.conf',
    notify => Service[THE_STUFF],
  }
}
\end{lstlisting}

\end{itemize}

\end{frame}
%%%%%%%%%%%%%%%%%%%%%%%%%%%%%%%%%%%%%%%%%%%%%%%%%%%%%%%%%%%%%%%%%%%%%%%%%%%%%%
%%%%%%%%%%%%%%%%%%%%%%%%%%%%%%%%%%%%%%%%%%%%%%%%%%%%%%%%%%%%%%%%%%%%%%%%%%%%%%
\section{Managing Puppet with Git}
\begin{frame}[fragile]
\frametitle{Managing Puppet with  Git}
\begin{itemize}
\item Create and commit git repository
\lstset{language=shell}
\begin{lstlisting}[escapechar=&]
$ cd ~/puppet
$ git init
$ git add .
$ git commit -a -m "Initial commit"
\end{lstlisting}

\end{itemize}

\end{frame}
%%%%%%%%%%%%%%%%%%%%%%%%%%%%%%%%%%%%%%%%%%%%%%%%%%%%%%%%%%%%%%%%%%%%%%%%%%%%%%
%%%%%%%%%%%%%%%%%%%%%%%%%%%%%%%%%%%%%%%%%%%%%%%%%%%%%%%%%%%%%%%%%%%%%%%%%%%%%%
\begin{frame}[fragile]
\frametitle{Distributing Puppet Manifests}
\begin{itemize}
\item To manage several servers at once, we need to distribute
the Puppet manifests to each machine so that they can be applied
\item There are several ways to do this, and Puppet has a built-in server capability (Puppetmaster) or using Git to distribute the manifests
\begin{itemize}
\item Reliability. You will still be able to run
Puppet on all your client machines and push changes to them using Git
\item Scalability. You can keep on adding machines indefinitely, and each one looks after itself. By contrast, using
a Puppetmaster moves all the workload of compiling manifests from the client machine to a
single server, which places heavy demands on that server as the network grows
\item Simplicity (``what's simple to you depends on what you already know''
\end{itemize}

\end{itemize}

\end{frame}
%%%%%%%%%%%%%%%%%%%%%%%%%%%%%%%%%%%%%%%%%%%%%%%%%%%%%%%%%%%%%%%%%%%%%%%%%%%%%%
%%%%%%%%%%%%%%%%%%%%%%%%%%%%%%%%%%%%%%%%%%%%%%%%%%%%%%%%%%%%%%%%%%%%%%%%%%%%%%
\begin{frame}[fragile, allowframebreaks]
\frametitle{Creating a master Git repo}

\begin{itemize}
 \item We're going to make a copy of our existing Puppet repo, which we can then clone on
a new machine.
\begin{enumerate}
\item Create a directory to hold the repo
\lstset{language=shell}
\begin{lstlisting}[escapechar=&]
$ sudo mkdir /var/git
\end{lstlisting}

\item Clone the repo, using the --bare flag:
\lstset{language=shell}
\begin{lstlisting}[escapechar=&]
$ cd /var/git
$ sudo git clone --bare ~/puppet
&\outputcommand{Cloning into bare repository 'puppet.git'...
done.}&
\end{lstlisting}

\item Now create a git user that will own the master repo and control access to it:
\lstset{language=shell}
\begin{lstlisting}[escapechar=&]
$ sudo adduser -d /var/git git
$ sudo chown -R git:git /var/git
\end{lstlisting}

\item Just to verify that these steps have worked, check out a temporary clone of the
master repo:
\lstset{language=shell}
\begin{lstlisting}[escapechar=&]
$ cd /tmp
$ git clone /var/git/puppet.git
\end{lstlisting}

\lstset{language=shell}
\begin{lstlisting}[escapechar=&]
$ ls puppet
&\outputcommand{manifests modules}&
$ rm -rf puppet
\end{lstlisting}

\item Now create a secure shell (SSH) keypair for the git user so that it can log in
from remote machines to clone and update the Git repo. When prompted for a
passphrase, just hit Enter.
\lstset{language=shell}
\begin{lstlisting}[escapechar=&]
$ sudo su - git
$ ssh-keygen
&\outputcommand{A lot of questions and stuff \dots }&
\end{lstlisting}

\item Create an authorized\_keys file for git containing the public key you just generated:
\begin{lstlisting}[escapechar=&]
$ cd .ssh
$ ls
$ cp id_rsa.pub authorized_keys
\end{lstlisting}

\item You should now be able to log into the git account via SSH using this key:
\begin{lstlisting}[escapechar=&]
$ ssh git@localhost
&\outputcommand{The authenticity of host 'localhost (::1)' can't be established.\\
RSA key fingerprint is 4f:f8:8f:54:82:35:84:7f:c5:d3:89:53:2a:bc:a4:ad.\\
Are you sure you want to continue connecting (yes/no)? yes\\
Warning: Permanently added 'localhost' (RSA) to the list of known hosts}
\end{lstlisting}

\end{enumerate}
\end{itemize}


\end{frame}
%%%%%%%%%%%%%%%%%%%%%%%%%%%%%%%%%%%%%%%%%%%%%%%%%%%%%%%%%%%%%%%%%%%%%%%%%%%%%%
%%%%%%%%%%%%%%%%%%%%%%%%%%%%%%%%%%%%%%%%%%%%%%%%%%%%%%%%%%%%%%%%%%%%%%%%%%%%%%
\begin{frame}[fragile]
\frametitle{Cloning the repo a new a machine}
\begin{itemize}
\item You'll need a second machine similar to the one you have been using so far (either a cloud
instance (EC2), a Vagrant VM, or a physical machine, whichever is convenient)
\item Install Puppet and its dependencies
\end{itemize}

\end{frame}
%%%%%%%%%%%%%%%%%%%%%%%%%%%%%%%%%%%%%%%%%%%%%%%%%%%%%%%%%%%%%%%%%%%%%%%%%%%%%%
%%%%%%%%%%%%%%%%%%%%%%%%%%%%%%%%%%%%%%%%%%%%%%%%%%%%%%%%%%%%%%%%%%%%%%%%%%%%%%
\end{document}

