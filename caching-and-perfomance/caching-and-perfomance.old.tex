\documentclass{beamer}
\usepackage[british,spanish]{babel}
\usepackage[utf8]{inputenc}

\usepackage{hyperref}
%\hypersetup{colorlinks=false,linkbordercolor=red,linkcolor=green,pdfborderstyle={/S/U/W 1}}

\usepackage{multirow}

\usepackage{textcomp}

\usepackage{listings}
\lstloadlanguages{Ruby}
\usepackage{cancel}

\usepackage{adjustbox}
\usepackage{lstcustom}

\usepackage{amsmath}

\usepackage{color}
\definecolor{light-gray}{gray}{0.80}
\definecolor{lstbackgroundshellcolor}{named}{light-gray}

\usepackage{tikz}
\newcommand*\circled[1]{\tikz[baseline=(char.base)]{
            \node[shape=circle,draw,inner sep=2pt] (char) {#1};}}

\usepackage[normalem]{ulem}


%\usepackage[acronym,xindy,toc]{glossaries}

\usepackage[acronym,xindy,toc]{glossaries}
\makeglossaries
%\usepackage[xindy]{imakeidx}
%\makeindex

\newcommand{\specialcell}[2][c]{%
  \begin{tabular}[#1]{@{}c@{}}#2\end{tabular}}
\newcommand{\comment}[2]{#2}

\newcommand{\commandinline}[1]{\lstinline[basicstyle=\small\lstfontfamily]{#1}}
\newcommand{\outputcommand}[1]{\color{darkgreen}{#1}}

\graphicspath{ {./images/} }

\title{Caching and Perfomance}
%\subtitle[short subtitle]{long subtitle}
\author[C. Cuenca, F. Quintana]{Carmelo Cuenca-Hernández and Francisca Quintana-Domínguez}
%\institute{Escuela Universitaria de Informática}
%\date[04/2013]{Abril - 2013}
\date{}
\titlegraphic{\includegraphics[width=0.5 \textwidth]{./images/logo_ulpgc_version_horizontal_rgb.eps}}



\pgfdeclareimage[width=2.0\baselineskip]{ulpgc-logo}{./images/logosimbolo_secundario_version_vertical}
\setbeamertemplate{footline}{\raisebox{-2ex}{\pgfuseimage{ulpgc-logo}}
  \usebeamerfont{date in head/foot}\insertshortdate{}\hfill
  \usebeamertemplate{navigation symbols}\hfill
  \insertframenumber{}/\inserttotalframenumber}
\setbeamertemplate{sidebar right}{}


\usetheme{Antibes}
%\usetheme{Berlin}

%\usetheme{Warsaw}
%\usecolortheme{albatross}

\selectlanguage{british}



\begin{document}


\begin{frame}
	\titlepage
\end{frame}


\section*{Outline}
\begin{frame}[fragile, allowframebreaks]
  \frametitle{Outline}
  %\tableofcontents%[part=1,pausesections]
  %\tableofcontents[currentsection,currentsubsection, sectionstyle=show] 
  \tableofcontents[currentsection,sectionstyle=show,hideothersubsections]
\end{frame}

%%%%%%%%%%%%%%%%%%%%%%%%%%%%%%%%%%%%%%%%%%%%%%%%%%%%%%%%%%%%%%%%%%%%%%%%%%%%%%
%\newacronym{<label>}{<abbrv>}{<full>}
%\glsreset{<label>}
%\glsresetall
%\acrlong{<label>}
%\acrfull{<label>}
%\acrshort{<label>}
%\input{../glossary}

\newacronym{ror}{RoR}{{\href{http://rubyonrails.org/}{Ruby on Rails}}}
\newacronym{rds}{RDS}{Relational Database Service}
\newacronym{rvm}{RVM}{{\href{https://rvm.io/}{Ruby Version Manager}}}
%%%%%%%%%%%%%%%%%%%%%%%%%%%%%%%%%%%%%%%%%%%%%%%%%%%%%%%%%%%%%%%%%%%%%%%%%%%%%%
\section{Sample App}
\begin{frame}[fragile, allowframebreaks]
\frametitle{Sample App}
\begin{itemize}
\item 
\end{itemize}

\begin{tabular}{|l|l|} \hline
Controller & Action \\ \hline
\texttt{static\_pages} & \texttt{home}, \texttt{help}, \texttt{about}, \texttt{contact} \\ \hline
\texttt{users} & \specialcell[l]{\texttt{new}, \texttt{show}, \texttt{index}, \texttt{create}, \texttt{edit} \\ \texttt{update}, \texttt{following}, \texttt{followers}}\\ \hline
\texttt{sessions} & \texttt{new}, \texttt{create}, \texttt{destroy} \\ \hline
\texttt{micropost} & \texttt{create}, \texttt{destroy}\\ \hline
\texttt{relationships} & \texttt{create}, \texttt{destroy}\\ \hline
\end{tabular}



\end{frame}
%%%%%%%%%%%%%%%%%%%%%%%%%%%%%%%%%%%%%%%%%%%%%%%%%%%%%%%%%%%%%%%%%%%%%%%%%%%%%%
%%%%%%%%%%%%%%%%%%%%%%%%%%%%%%%%%%%%%%%%%%%%%%%%%%%%%%%%%%%%%%%%%%%%%%%%%%%%%%
\section{View Caching}
\begin{frame}[fragile, allowframebreaks]
\frametitle{View Caching}
\begin{itemize}
\item Rails application could be very responsive and scalable (YellowPages (top 40), GitHub,Twitter, Groupon, SlideShare (60 millions monthly visitors))
\item View caching lets you specify that anything from entire pages down to fragments of the page should be captured to ``disk'' as HTML files and sent along by your web server on future request with minimal involvement from Rails itself
\begin{enumerate}
\item Page caching. The output of an entire controller is cached to disk, with no further involvement by the Rails dispatcher
\item Action caching. The output of an entire controller action is cached to disk, but he Rails dispatcher is still involved in subsequent requests, and controller filters are executed
\item Fragment Caching. Arbitrary bits and pieces of your page’s output can be cached to disk
to save the time of having to render them in the future
\end{enumerate}
\item ETag support means that in the best-case scenarios, it's not even necessary to send any content at all back to the browser, beyond a couple of HTTP headers

\item Page Caching has been removed from Rails 4 so we have to install a gem for that\dots
\lstset{language=shell}
\begin{lstlisting}[escapechar=!]
$ echo "gem 'actionpack-page_caching'" >> Gemfile
$ bundle install
\end{lstlisting}
\end{itemize}

\end{frame}
%%%%%%%%%%%%%%%%%%%%%%%%%%%%%%%%%%%%%%%%%%%%%%%%%%%%%%%%%%%%%%%%%%%%%%%%%%%%%%
\begin{frame}[fragile]
\frametitle{Caching in Development Mode?}
\begin{itemize}
\item If we want to play with caching during development, we’ll need to edit the following setting in the \texttt{config/environments/development.rb} file:

\lstset{language=shell}
\begin{lstlisting}
config.action_controller.perform_caching = false
\end{lstlisting}

\item Of course, remember to change it back before checking it back into your project repository,
or you might face some very confusing errors down the road
\end{itemize}

\end{frame}
%%%%%%%%%%%%%%%%%%%%%%%%%%%%%%%%%%%%%%%%%%%%%%%%%%%%%%%%%%%%%%%%%%%%%%%%%%%%%%
\subsection{Page Caching}
\begin{frame}[fragile]
\frametitle{Page Caching}
\begin{itemize}
\item The simplest form of caching is page caching, triggered by use of the \texttt{caches\_page}
macro-style method in a controller
\item It tells Rails to capture the entire output of the
request to disk so that it is served up directly by the web server on subsequent requests
without the involvement of the dispatcher
\item Nothing will be logged to the Rails log, nor
will controller filters be triggered—absolutely nothing to do with Rails will happen, just
like the static HTML files in your project’s public directory
\end{itemize}
\end{frame}
%%%%%%%%%%%%%%%%%%%%%%%%%%%%%%%%%%%%%%%%%%%%%%%%%%%%%%%%%%%%%%%%%%%%%%%%%%%%%%
\begin{frame}[fragile, allowframebreaks]
\frametitle{Page Caching - Home Page}
\begin{itemize}
\item Run Rails server and  visit the \texttt{sample\_app home page} in order to get measures of ``views time'' and ``active record time''
\lstset{language=shell}
\begin{lstlisting}[escapechar=!]
$ curl http://localhost:3000
!\outputcommand{Started GET "/" for 127.0.0.1 at 2014-04-02 12:43:15 +0100\\
Processing by StaticPagesController\#home as */*\\
\vdots
  Rendered layouts/\_header.html.erb (1.8ms)\\
  Rendered layouts/\_footer.html.erb (0.6ms)\\
Completed 200 OK in 39ms (Views: 26.4ms | ActiveRecord: 0.5ms)\\
}!
\end{lstlisting}

\item To add page caching to the home page it is necessary to modify the files \texttt{config/routes.rb} and \texttt{app/controllers/static\_pages\_controller.rb}. 

\lstset{language=Ruby, style=eclipse}
\begin{lstlisting}[escapechar=!]
# config/routes.rb
!\vdots!
  match '/public',  to: 'static_pages#public', via: 'get'
  match '/', to: 'static_pages#home', via: 'get'
!\vdots!
\end{lstlisting}

\lstset{language=Ruby, style=eclipse}
\begin{lstlisting}
# app/controllers/static_pages_controller.rb
class StaticPagesController < ApplicationController
  before_filter :check_logged_in, only: [:home]

  caches_page :public

  def public
    render :home
  end

  def home
    @micropost = current_user.microposts.build
    @feed_items = current_user.feed.paginate(page: params[:page])
  end

  def help
  end

  def about
  end

  def contact
  end

  private

  def check_logged_in
    redirect_to public_path unless signed_in?
  end

end
\end{lstlisting}

\item Get new measures of ``views time'' and ``active record time''

\lstset{language=shell}
\begin{lstlisting}[escapechar=!]
$ $ curl http://localhost:3000
!\outputcommand{Started GET "/" for 127.0.0.1 at 2014-04-02 15:03:42 +0100\\
\vdots
Completed 302 Found in 19ms (ActiveRecord: 0.5ms)\\
}!
\end{lstlisting}

\item The web server wrote the page in the location \texttt{public/static\_pages/public.html}

\item Run the rspec test to check the color of the refactoring

\lstset{language=shell}
\begin{lstlisting}
$ spec rspec/
\end{lstlisting}
\end{itemize}


\end{frame}
%%%%%%%%%%%%%%%%%%%%%%%%%%%%%%%%%%%%%%%%%%%%%%%%%%%%%%%%%%%%%%%%%%%%%%%%%%%%%%

\section{Homeworks}
\begin{frame}[fragile]
\frametitle{Homeworks}
\begin{itemize}
\item 
\end{itemize}

\end{frame}
\end{document}
