\documentclass{beamer}
\usepackage[british,spanish]{babel}
\usepackage[utf8]{inputenc}

\usepackage{hyperref}
%\hypersetup{colorlinks=false,linkbordercolor=red,linkcolor=green,pdfborderstyle={/S/U/W 1}}

\usepackage{multirow}

\usepackage{textcomp}

\usepackage{listings}
\lstloadlanguages{Ruby}
\usepackage{cancel}

\usepackage{adjustbox}
\usepackage{lstcustom}

\usepackage{amsmath}

\usepackage{color}
\definecolor{light-gray}{gray}{0.80}
\definecolor{lstbackgroundshellcolor}{named}{light-gray}

\usepackage{tikz}
\newcommand*\circled[1]{\tikz[baseline=(char.base)]{
            \node[shape=circle,draw,inner sep=2pt] (char) {#1};}}

\usepackage[normalem]{ulem}


%\usepackage[acronym,xindy,toc]{glossaries}

\usepackage[acronym,xindy,toc]{glossaries}
\makeglossaries
%\usepackage[xindy]{imakeidx}
%\makeindex

\newcommand{\specialcell}[2][c]{%
  \begin{tabular}[#1]{@{}c@{}}#2\end{tabular}}
\newcommand{\comment}[2]{#2}

\newcommand{\commandinline}[1]{\lstinline[basicstyle=\small\lstfontfamily]{#1}}
\newcommand{\outputcommand}[1]{\color{darkgreen}{#1}}

\graphicspath{ {./images/} }

\title{Caching and Perfomance}
%\subtitle[short subtitle]{long subtitle}
\author[C. Cuenca, F. Quintana]{Carmelo Cuenca-Hernández and Francisca Quintana-Domínguez}
%\institute{Escuela Universitaria de Informática}
%\date[04/2013]{Abril - 2013}
\date{}
\titlegraphic{\includegraphics[width=0.5 \textwidth]{./images/logo_ulpgc_version_horizontal_rgb.eps}}



\pgfdeclareimage[width=2.0\baselineskip]{ulpgc-logo}{./images/logosimbolo_secundario_version_vertical}
\setbeamertemplate{footline}{\raisebox{-2ex}{\pgfuseimage{ulpgc-logo}}
  \usebeamerfont{date in head/foot}\insertshortdate{}\hfill
  \usebeamertemplate{navigation symbols}\hfill
  \insertframenumber{}/\inserttotalframenumber}
\setbeamertemplate{sidebar right}{}


\usetheme{Antibes}
%\usetheme{Berlin}

%\usetheme{Warsaw}
%\usecolortheme{albatross}

\selectlanguage{british}



\begin{document}


\begin{frame}
	\titlepage
\end{frame}


\section*{Outline}
\begin{frame}[fragile, allowframebreaks]
  \frametitle{Outline}
  %\tableofcontents%[part=1,pausesections]
  %\tableofcontents[currentsection,currentsubsection, sectionstyle=show] 
  \tableofcontents[currentsection,sectionstyle=show,hideothersubsections]
\end{frame}

%%%%%%%%%%%%%%%%%%%%%%%%%%%%%%%%%%%%%%%%%%%%%%%%%%%%%%%%%%%%%%%%%%%%%%%%%%%%%%
%\newacronym{<label>}{<abbrv>}{<full>}
%\glsreset{<label>}
%\glsresetall
%\acrlong{<label>}
%\acrfull{<label>}
%\acrshort{<label>}
%\input{../glossary}

\newacronym{ror}{RoR}{{\href{http://rubyonrails.org/}{Ruby on Rails}}}
\newacronym{rds}{RDS}{Relational Database Service}
\newacronym{rvm}{RVM}{{\href{https://rvm.io/}{Ruby Version Manager}}}


%%%%%%%%%%%%%%%%%%%%%%%%%%%%%%%%%%%%%%%%%%%%%%%%%%%%%%%%%%%%%%%%%%%%%%%%%%%%%%
\begin{frame}[fragile, allowframebreaks]
\frametitle{Caching and Perfomance}
\begin{itemize}
\item Rails application could be very responsive and scalable (YellowPages (top 40), GitHub,Twitter, Groupon, SlideShare (60 millions monthly visitors))
\item HTTP caching means that in the best-case scenarios, it's not even necessary to send any content at all back to the browser, beyond a couple of HTTP headers

\item View caching lets you specify that anything from entire pages down to fragments of the page should be captured to ``disk'' as HTML files and sent along by your web server on future request with minimal involvement from Rails itself
\begin{enumerate}
\item Page caching. The output of an entire controller is cached to disk, with no further involvement by the Rails dispatcher
\item Action caching. The output of an entire controller action is cached to disk, but he Rails dispatcher is still involved in subsequent requests, and controller filters are executed
\item Fragment Caching. Arbitrary bits and pieces of your page’s output can be cached to disk
to save the time of having to render them in the future
\end{enumerate}

\item Page Caching has been removed from Rails 4 so we have to install a gem for that\dots
\lstset{language=shell}
\begin{lstlisting}[escapechar=!]
$ echo "gem 'actionpack-page_caching'" >> Gemfile
$ bundle install
\end{lstlisting}
\end{itemize}

\end{frame}
%%%%%%%%%%%%%%%%%%%%%%%%%%%%%%%%%%%%%%%%%%%%%%%%%%%%%%%%%%%%%%%%%%%%%%%%%%%%%%
\section{A Simple Example Application}
\begin{frame}[fragile, allowframebreaks]
\frametitle{A Simple Example Application}
\begin{itemize}
\item Create the new Rails app

\lstset{language=shell}
\begin{lstlisting}[escapechar=!]
$ rvm use 2.0.0@phone_book_app --create
!\outputcommand{[\dots]}!
$ gem install rails --vesion=4.0.4
!\outputcommand{[\dots]}!
$ rails new phone_book_app --skip-test-unit
!\outputcommand{[\dots]}!
$ cd new phone_book_app/
$ echo '2.0.0' > .ruby-version; echo 'phone_book_app' > .ruby-gemset
\end{lstlisting}

\item Add and install some gems

\lstset{language=Ruby, style=eclipse}
\begin{lstlisting}[escapechar=!]
# Gemfile
!\vdots!
gem 'faker'

group :production do
  gem 'pg', '0.15.1'
  gem 'rails_12factor', '0.0.2'
end
\end{lstlisting}

\lstset{language=shell}
\begin{lstlisting}[escapechar=!]
$ bundle install --without production
!\outputcommand{[\dots]}!
\end{lstlisting}

\item Generate two scaffolds
\lstset{language=shell}
\begin{lstlisting}[escapechar=!]
$ rails generate scaffold company name --no--test-framework
!\outputcommand{[\dots]}!
$ rails generate scaffold employee company:references last_name first_name phone_number --no-test-framework
!\outputcommand{[\dots]}!
$ rake db:migrate
!\outputcommand{[\dots]}!
\end{lstlisting}

\item Insert a few rules in the two models
\lstset{language=Ruby, style=eclipse}
\begin{lstlisting}
# app/models/company.rb
class Company < ActiveRecord::Base
  validates :name, presence: true, uniqueness: true
  has_many :employees, dependent: :destroy

  def to_s
    name
  end
end
\end{lstlisting}

\lstset{language=Ruby, style=eclipse}
\begin{lstlisting}
# app/models/employee.rb
class Employee < ActiveRecord::Base
  belongs_to :company, touch: true

  validates :first_name, presence: true
  validates :last_name, presence: true
  validates :company, presence: true

  def to_s
    "#{first_name} #{last_name}"
  end
end
\end{lstlisting}
 
\item Change the index company view file in order to list the total number of employees in each company
\lstset{language=Ruby, style=eclipse}
\begin{lstlisting}[escapechar=&]
# app/views/companies/show.html.erb
<h1>Listing companies</h1>

<table>
  <thead>
    <tr>
      <th>Name</th>
      <th>Number of employees</th> <!-- &\circled{1}& -->
      <th></th>
      <th></th>
      <th></th>
    </tr>
  </thead>

  <tbody>
    <% @companies.each do |company| %>
      <tr>
        <td><%= company.name %></td>
        <td><%= company.employees.count %></td> <!-- &\circled{2}& -->
        <td><%= link_to 'Show', company %></td>
        <td><%= link_to 'Edit', edit_company_path(company) %></td>
        <td><%= link_to 'Destroy', company, method: :delete, data: { confirm: 'Are you sure?' } %></td>
      </tr>
    <% end %>
  </tbody>
</table>

<br>

<%= link_to 'New Company', new_company_path %>

\end{lstlisting}

\item Set the root url
\lstset{language=Ruby, style=eclipse}
\begin{lstlisting}[escapechar=&]
# config/routes.rb
PhoneBookApp::Application.routes.draw do
  resources :employees

  resources :companies


  root 'companies#index' # &\circled{1}&
end
\end{lstlisting}

\item Populate the database with the Faker gem
\lstset{language=Ruby, style=eclipse}
\begin{lstlisting}[escapechar=&]
# db/seeds.rb

64.times do
  company = Company.new(name: Faker::Company.name)
  if company.save ETag support 
    SecureRandom.random_number(128).times do
      company.employees.create(first_name: Faker::Name.first_name,
                               last_name: Faker::Name.last_name,
                               phone_number: Faker::PhoneNumber.phone_number)
    end
  end
end
\end{lstlisting}

\lstset{language=shell}
\begin{lstlisting}[escapechar=!]
$ rake db:seed
!\outputcommand{[\dots]}!
\end{lstlisting}

\item Start the server and list all companies (index view) and show one (show view)
\lstset{language=shell}
\begin{lstlisting}[escapechar=!]
$ curl http://localhost:3000
!\outputcommand{[\dots]\\
Rendered companies/index.html.erb within layouts/application (112.2ms)\\
Completed 200 OK in {\color{red}189ms (Views: 153.6ms | ActiveRecord: 7.9ms)}}!
$ curl http://localhost:3000/companies/1/
!\outputcommand{[\dots]\\
Rendered companies/show.html.erb within layouts/application (25.7ms)\\
Completed 200 OK in {\color{red}130ms (Views: 76.7ms | ActiveRecord: 2.3ms)}}!
\end{lstlisting}
\end{itemize}
\end{frame}
%%%%%%%%%%%%%%%%%%%%%%%%%%%%%%%%%%%%%%%%%%%%%%%%%%%%%%%%%%%%%%%%%%%%%%%%%%%%%%
%%%%%%%%%%%%%%%%%%%%%%%%%%%%%%%%%%%%%%%%%%%%%%%%%%%%%%%%%%%%%%%%%%%%%%%%%%%%%%
\section{HTTP Caching}
\begin{frame}[fragile]
\frametitle{HTTP Caching}
\begin{itemize}
\item HTTP Caching attempts to reuse already loaded web pages or files
\item static contents, dynamic contents
\end{itemize}
\end{frame}


\subsection{Last Modified}
\begin{frame}[fragile, allowframebreaks]
\frametitle{Last Modified}
\begin{itemize}
\item The web browser knows when it has downloaded a web page and the placed it into the cache. It can pass this infomation to the web server in an \texttt{If-Modified-Since:} header. The web server can then compare this information to the corresponding file and either deliver a newer version or return an \texttt{HTTP 304 Not Modified} code as response

\item Edit the \texttt{show} method in \texttt{app/controllers/companies\_controller.rb}

\lstset{language=Ruby, style=eclipse}
\begin{lstlisting}[escapechar=&]
# app/controllers/companies_controller.rb
&\vdots&
  def show
    fresh_when last_modified: @company.updated_at
  end
&\vdots&
\end{lstlisting}

\item This will render the show template if the request isn’t sending an \texttt{If-Modified-Since} header and just a \texttt{304 Not Modified} response if there’s a match

\lstset{language=shell}
\begin{lstlisting}[escapechar=!]
$ curl -I http://localhost:3000/companies/1/
!\outputcommand{HTTP/1.1 200 OK\\
[\dots]\\
Last-Modified: Thu, 03 Apr 2014 12:01:34 GMT\\
}!
$ curl -I http://localhost:3000/companies/1 --header 'If-Modified-Since: Thu, 03 Apr 2014 18:06:27 GMT'
!\outputcommand{HTTP/1.1 {\color{red}304 Not Modified}\\
[\dots]}!
\end{lstlisting}
\end{itemize}
\end{frame}
%%%%%%%%%%%%%%%%%%%%%%%%%%%%%%%%%%%%%%%%%%%%%%%%%%%%%%%%%%%%%%%%%%%%%%%%%%%%%%
\subsection{Etag}
\begin{frame}[fragile, allowframebreaks]
\frametitle{Etag}
\begin{itemize}
\item Sometimes the \texttt{update\_at} field of a particular object is not meaningful on its own. For example, if you have a web page where users can log in and this page then generates web page contents based on a role model, it can happen that an user as admin is able to see an Edit link that is not displayed to user B as normal user. In such a scenario, the Last-Modified header does not help
\item The Etag is generated by the web server and delivered when the web page is first visited. If the user visits the same URL again, the browser can the check if the corresponding web page has changed by sending a \texttt{If-None-Match:} query to the web server


\item Edit the \texttt{index} and \texttt{show} methods in \texttt{app/controllers/companies\_controller.rb}
\lstset{language=Ruby, style=eclipse}
\begin{lstlisting}[escapechar=&]
# app/controllers/companies_controller.rb
&\vdots&
  def index
    @companies = Company.all
    fresh_when etag: @companies
  end

  # GET /companies/1
  # GET /companies/1.json
  def show
    fresh_when etag: @company
  end
&\vdots&
\end{lstlisting}
\item Rails automatically set a new CSRF (cross site request forgery) token for each new visitor of the website. Each new user of a web page gets a new etag for the same page. To ensure that the same users also get identical CSRF tokens, these are stored in a cookie by the web browser

\item Use \texttt{curl} to test your application

\begin{itemize}

\item With the parameter \texttt{-c cookies.txt}, curls saves all cookies in a file

\lstset{language=shell}
\begin{lstlisting}[escapechar=!]
$ curl -I http://localhost:3000/companies -c cookies.txt
!\outputcommand{HTTP/1.1 200 OK\\
[\dots]\\
Etag: "3f5571a88b47b9cd72b820d7678374b8"\\
}!
\end{lstlisting}

\item With the parameter \texttt{-b cookies.txt}, curls sends the cookies to the web server. We should get the same etag
\lstset{language=shell}
\begin{lstlisting}[escapechar=!]
$ curl -I http://localhost:3000/companies -b cookies.txt
!\outputcommand{HTTP/1.1 200 OK\\
[\dots]\\
Etag: "3f5571a88b47b9cd72b820d7678374b8"\\
}!
\end{lstlisting}

\item Use the etag to find out in the request with \texttt{If-None-Match} if the version which was cached is still up to date
\lstset{language=shell}
\begin{lstlisting}[escapechar=!]
$ curl -I http://localhost:3000/companies -b cookies.txt --header 'If-None-Match: "3f5571a88b47b9cd72b820d7678374b8"'
!\outputcommand{HTTP/1.1 {\color{red}304 Not Modified}\\
[\dots]\\
Etag: "3f5571a88b47b9cd72b820d7678374b8"\\
}!
\end{lstlisting}
\end{itemize}

\item Etag is not limitted to only one parameter. So you can  use with \texttt{current\_user} and other potencial parameters

\lstset{language=Ruby, style=eclipse}
\begin{lstlisting}[escapechar=&]
# app/controllers/companies_controller.rb
&\vdots&
  def index
    @companies = Company.all
    fresh_when(etag: [@companies, current_user.try :id])
  end

  # GET /companies/1
  # GET /companies/1.json
  def show
    fresh_when etag:([@company, current_user.try :id])
  end
&\vdots&
\end{lstlisting}

or DRY 

\lstset{language=Ruby, style=eclipse}
\begin{lstlisting}[escapechar=&]
# app/controllers/application_controller.rb
class ApplicationController < ActionController::Base
  # Prevent CSRF attacks by raising an exception.
  # For APIs, you may want to use :null_session instead.
  protect_from_forgery with: :exception

  etag {currrent_user.try :id}
end
\end{lstlisting}
\end{itemize}

\end{frame}
%%%%%%%%%%%%%%%%%%%%%%%%%%%%%%%%%%%%%%%%%%%%%%%%%%%%%%%%%%%%%%%%%%%%%%%%%%%%%%
\begin{frame}[fragile]
\frametitle{Cache-Control with time limit}
\begin{itemize}
\item You can take the optimization one step further by predicting the future. 

\lstset{language=Ruby, style=eclipse}
lstset{language=Ruby, style=eclipse}
\begin{lstlisting}[escapechar=&]
# app/controllers/companies_controller.rb
&\vdots&
  def show
    expires_in 2.minutes
    fresh_when etag:@company
  end
&\vdots&
\end{lstlisting}

\item Go to the web page and get the oputput
\lstset{language=shell}
\begin{lstlisting}[escapechar=!]
$ curl -I http://localhost:3000/companies/1
!\outputcommand{[\dots]\\
Cache-Control: {\color{red}max-age=120}, private\\
[\dots]\\
}!
\end{lstlisting}

\end{itemize}

\end{frame}
%%%%%%%%%%%%%%%%%%%%%%%%%%%%%%%%%%%%%%%%%%%%%%%%%%%%%%%%%%%%%%%%%%%%%%%%%%%%%%
\begin{frame}[fragile, allowframebreaks]
\subsection{Using proxies}
\frametitle{Using Proxies}
\begin{itemize}
\item There are many proxies that are often closer to the user. If our example was a publicly accessible phone book, then we could activate the free services of the proxies with the parameter \texttt{public: true} in \texttt{fresh\_when}

\lstset{language=Ruby, style=eclipse}
\begin{lstlisting}[escapechar=&]
# app/controllers/companies_controller.rb
&\vdots&
  def show
    fresh_when etag:@company, public: true
  end
&\vdots&
\end{lstlisting}

\item Go to the web page and get the oputput
\lstset{language=shell}
\begin{lstlisting}[escapechar=!]
$ curl -I http://localhost:3000/companies/1
!\outputcommand{[\dots]\\
Cache-Control: {\color{red}{public}}\\
[\dots]\\
}!
\end{lstlisting}

\item Using proxies always has to be done with great caution. For example CSRF tags and Flash messages should never end up in a public proxy. It is a good idea to make the output of \texttt{csrf\_meta\_tag} in the dafault \texttt{app/views/layouts/application.html.erb} layout dependent on the question whether the page may be cached publicly or not:
\lstset{language=Ruby, style=eclipse}
\begin{lstlisting}[escapechar=&]
# app/views/layouts/application.html.erb
<%= csrf_meta_tag unless response.cache_control[:public] %>
\end{lstlisting}
 
\end{itemize}
\end{frame}

%%%%%%%%%%%%%%%%%%%%%%%%%%%%%%%%%%%%%%%%%%%%%%%%%%%%%%%%%%%%%%%%%%%%%%%%%%%%%%

\section{Rails Caching}
\begin{frame}[fragile]
\frametitle{Caching in Development Mode?}
\begin{itemize}
\item If we want to play with caching during development, we’ll need to edit the following setting in the \texttt{config/environments/development.rb} file:

\lstset{language=shell}
\begin{lstlisting}[escapechar=!]
#config/environments/development.rb
!\vdots!
config.action_controller.perform_caching = true
!\vdots!
\end{lstlisting}

\item Of course, remember to change it back before checking it back into your project repository,
or you might face some very confusing errors down the road
\end{itemize}

\end{frame}
%%%%%%%%%%%%%%%%%%%%%%%%%%%%%%%%%%%%%%%%%%%%%%%%%%%%%%%%%%%%%%%%%%%%%%%%%%%%%%
%%%%%%%%%%%%%%%%%%%%%%%%%%%%%%%%%%%%%%%%%%%%%%%%%%%%%%%%%%%%%%%%%%%%%%%%%%%%%

\subsection{Fragment Caching}
\begin{frame}[fragile, , allowframebreaks]
\frametitle{Fragment Caching}
\begin{itemize}
\item With fragment caching you can cache individual parts of a view. The advantages once again are a reduction of server load and faster web page generation, which means increased usubility

\item Reset the \texttt{index} and \texttt{show} methods in \texttt{app/controllers/companies\_controller.rb}
\lstset{language=Ruby, style=eclipse}
\begin{lstlisting}[escapechar=&]
def index
    @companies = Company.all
  end

  # GET /companies/1
  # GET /companies/1.json
  def show
  end
\end{lstlisting}

\item Edit the file \texttt{app/views/companies/index.html.rb}
\lstset{language=Ruby, style=eclipse}
\begin{lstlisting}[escapechar=&]
# app/views/companies/index.html.rb
<h1>Listing companies</h1>

<% cache('companies') do %>  <!-- &\circled{1}& -->
  <table>
&\vdots&
  </table>
<% end %>  <!-- &\circled{2}& -->

<br>

<%= link_to 'New Company', new_company_path %>
\end{lstlisting}

\item Go twice to web server root url
\lstset{language=shell}
\begin{lstlisting}[escapechar=!]
$ curl http://localhost:3000/companies
!\outputcommand{Write fragment views/companies/a206e211f024c0985d3a493c599e70fb (0.9ms)\\
  Rendered companies/index.html.erb within layouts/application (99.2ms)\\
Completed 200 OK in {\color{red}552ms (Views: 520.7ms | ActiveRecord: 6.4ms)}
}!
$ curl http://localhost:3000/companies
!\outputcommand{Read fragment views/companies/a206e211f024c0985d3a493c599e70fb (0.3ms)\\
  Rendered companies/index.html.erb within layouts/application (6.1ms)\\
Completed 200 OK in {\color{red}17ms (Views: 16.2ms | ActiveRecord: 0.0ms)}
}!
\end{lstlisting}

\item It is possible to use Active record model as the cache key:
\lstset{language=Ruby, style=eclipse}
\begin{lstlisting}[escapechar=&]
# app/views/companies/index.html.rb
&\vdots&
<tbody>
  <% @companies.each do |company| %>
    <tr>
      <% cache(company) do %> # &&
        <td><%= company.name %></td>
        <td><%= company.employees.count %></td>
        <td><%= link_to 'Show', company %></td>
        <td><%= link_to 'Edit', edit_company_path(company) %></td>
        <td><%= link_to 'Destroy', company, method: :delete, data: { confirm: 'Are you sure?' } %></td>
      <% end %>
    </tr>
  <% end %>
</tbody>
&\vdots&
\end{lstlisting}

\end{itemize}

\end{frame}
%%%%%%%%%%%%%%%%%%%%%%%%%%%%%%%%%%%%%%%%%%%%%%%%%%%%%%%%%%%%%%%%%%%%%%%%%%%%%%
%%%%%%%%%%%%%%%%%%%%%%%%%%%%%%%%%%%%%%%%%%%%%%%%%%%%%%%%%%%%%%%%%%%%%%%%%%%%%%
\end{document}
%%%%%%%%%%%%%%%%%%%%%%%%%%%%%%%%%%%%%%%%%%%%%%%%%%%%%%%%%%%%%%%%%%%%%%%%%%%%%%
\subsection{Page Caching}
\begin{frame}[fragile]
\frametitle{Page Caching}
\begin{itemize}
\item The simplest form of caching is page caching, triggered by use of the \texttt{caches\_page}
macro-style method in a controller
\item It tells Rails to capture the entire output of the
request to disk so that it is served up directly by the web server on subsequent requests
without the involvement of the dispatcher
\item Nothing will be logged to the Rails log, nor
will controller filters be triggered—absolutely nothing to do with Rails will happen, just
like the static HTML files in your project’s public directory
\end{itemize}
\end{frame}
%%%%%%%%%%%%%%%%%%%%%%%%%%%%%%%%%%%%%%%%%%%%%%%%%%%%%%%%%%%%%%%%%%%%%%%%%%%%%%
\begin{frame}[fragile, allowframebreaks]
\frametitle{Page Caching - Home Page}
\begin{itemize}
\item Run Rails server and  visit the \texttt{sample\_app home page} in order to get measures of ``views time'' and ``active record time''
\lstset{language=shell}
\begin{lstlisting}[escapechar=!]
$ curl http://localhost:3000
!\outputcommand{Started GET "/" for 127.0.0.1 at 2014-04-02 12:43:15 +0100\\
Processing by StaticPagesController\#home as */*\\
\vdots
  Rendered layouts/\_header.html.erb (1.8ms)\\
  Rendered layouts/\_footer.html.erb (0.6ms)\\
Completed 200 OK in 39ms (Views: 26.4ms | ActiveRecord: 0.5ms)\\
}!
\end{lstlisting}

\item To add page caching to the home page it is necessary to modify the files \texttt{config/routes.rb} and \texttt{app/controllers/static\_pages\_controller.rb}. 

\lstset{language=Ruby, style=eclipse}
\begin{lstlisting}[escapechar=!]
# config/routes.rb
!\vdots!
  match '/public',  to: 'static_pages#public', via: 'get'
  match '/', to: 'static_pages#home', via: 'get'
!\vdots!
\end{lstlisting}

\lstset{language=Ruby, style=eclipse}
\begin{lstlisting}
# app/controllers/static_pages_controller.rb
class StaticPagesController < ApplicationController
  before_filter :check_logged_in, only: [:home]

  caches_page :public

  def public
    render :home
  end

  def home
    @micropost = current_user.microposts.build
    @feed_items = current_user.feed.paginate(page: params[:page])
  end

  def help
  end

  def about
  end

  def contact
  end

  private

  def check_logged_in
    redirect_to public_path unless signed_in?
  end

end
\end{lstlisting}

\item Get new measures of ``views time'' and ``active record time''

\lstset{language=shell}
\begin{lstlisting}[escapechar=!]
$ $ curl http://localhost:3000
!\outputcommand{Started GET "/" for 127.0.0.1 at 2014-04-02 15:03:42 +0100\\
\vdots
Completed 302 Found in 19ms (ActiveRecord: 0.5ms)\\
}!
\end{lstlisting}

\item The web server wrote the page in the location \texttt{public/static\_pages/public.html}

\item Run the rspec test to check the color of the refactoring

\lstset{language=shell}
\begin{lstlisting}
$ spec rspec/
\end{lstlisting}
\end{itemize}


\end{frame}
%%%%%%%%%%%%%%%%%%%%%%%%%%%%%%%%%%%%%%%%%%%%%%%%%%%%%%%%%%%%%%%%%%%%%%%%%%%%%%

\section{Homeworks}
\begin{frame}[fragile]
\frametitle{Homeworks}
\begin{itemize}
\item 
\end{itemize}

\end{frame}
\end{document}
