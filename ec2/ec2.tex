\documentclass{beamer}
\usepackage[british,spanish]{babel}
\usepackage[utf8]{inputenc}

\usepackage{hyperref}
%\hypersetup{colorlinks=false,linkbordercolor=red,linkcolor=green,pdfborderstyle={/S/U/W 1}}

\usepackage{multirow}

\usepackage{textcomp}

\usepackage{listings}
\lstloadlanguages{Ruby}
\usepackage{cancel}

\usepackage{adjustbox}
\usepackage{lstcustom}

\usepackage{amsmath}

\usepackage{color}
\definecolor{light-gray}{gray}{0.80}
\definecolor{lstbackgroundshellcolor}{named}{light-gray}

\usepackage{tikz}
\newcommand*\circled[1]{\tikz[baseline=(char.base)]{
            \node[shape=circle,draw,inner sep=2pt] (char) {#1};}}

\usepackage[normalem]{ulem}

%\usepackage[acronym,xindy,toc]{glossaries}

\usepackage[acronym,xindy,toc]{glossaries}
\makeglossaries
%\usepackage[xindy]{imakeidx}
%\makeindex


\newcommand{\comment}[2]{#2}

\newcommand{\commandinline}[1]{\lstinline[basicstyle=\small\lstfontfamily]{#1}}
\newcommand{\outputcommand}[1]{\color{darkgreen}{#1}}

\graphicspath{ {./images/} }

\title{Using the EC2 API}
%\subtitle[short subtitle]{long subtitle}
\author[C. Cuenca, F. Quintana]{Carmelo Cuenca-Hernández and Francisca Quintana-Domínguez}
%\institute{Escuela Universitaria de Informática}
%\date[04/2013]{Abril - 2013}
\date{}
\titlegraphic{\includegraphics[width=0.5 \textwidth]{./images/logo_ulpgc_version_horizontal_rgb.eps}}



\pgfdeclareimage[width=2.0\baselineskip]{ulpgc-logo}{./images/logosimbolo_secundario_version_vertical}
\setbeamertemplate{footline}{\raisebox{-2ex}{\pgfuseimage{ulpgc-logo}}
  \usebeamerfont{date in head/foot}\insertshortdate{}\hfill
  \usebeamertemplate{navigation symbols}\hfill
  \insertframenumber{}/\inserttotalframenumber}
\setbeamertemplate{sidebar right}{}


\usetheme{Antibes}
%\usetheme{Berlin}

%\usetheme{Warsaw}
%\usecolortheme{albatross}

\selectlanguage{british}



\begin{document}


\begin{frame}
	\titlepage
\end{frame}


\section*{Outline}
\begin{frame}[fragile, allowframebreaks]
  \frametitle{Outline}
  %\tableofcontents%[part=1,pausesections]
  %\tableofcontents[currentsection,currentsubsection, sectionstyle=show] 
  \tableofcontents[currentsection,sectionstyle=show,hideothersubsections]
\end{frame}

%%%%%%%%%%%%%%%%%%%%%%%%%%%%%%%%%%%%%%%%%%%%%%%%%%%%%%%%%%%%%%%%%%%%%%%%%%%%%%
%\newacronym{<label>}{<abbrv>}{<full>}
%\glsreset{<label>}
%\glsresetall
%\acrlong{<label>}
%\acrfull{<label>}
%\acrshort{<label>}
%\input{../glossary}

\newacronym{aws}{AWS}{Amazon Web Services}
\newacronym{css}{CSS}{cascading style sheets}
\newacronym{ebs}{EBS}{Elastic Block Storage}
\newacronym{ec2}{EC2}{Amazon Elastic Compute Cloud}
\newacronym{elb}{ELB}{Elastic Load Balancing}
\newacronym{ror}{RoR}{{\href{http://rubyonrails.org/}{Ruby on Rails}}}
\newacronym{rds}{RDS}{Relational Database Service}
\newacronym{rvm}{RVM}{{\href{https://rvm.io/}{Ruby Version Manager}}}
\newacronym{s3}{S3}{Simple Storage Service}
\newacronym{sqs}{SQS}{Amazon Simple Queue Service}
%%%%%%%%%%%%%%%%%%%%%%%%%%%%%%%%%%%%%%%%%%%%%%%%%%%%%%%%%%%%%%%%%%%%%%%%%%%%%%
\section{Using the EC2 API}
\begin{frame}[fragile]
\frametitle{Using the EC2 API}
\begin{itemize}
\item Key to the concept of the programmable data center is the fact that every resource
can be manipulated by an external program. In just over 50 lines (including comments and error
checking) the program does the following:
\begin{enumerate}
\item launches an EC2 instance
\item waits for the instance to transition to the ``running'' state
\item allocates a public IP address
\item attaches the IP address to the EC2 instance
\item allocates a 1GB EBS volume
\item attaches the volume to the EC2 instance
\end{enumerate}
\end{itemize}

\end{frame}
%%%%%%%%%%%%%%%%%%%%%%%%%%%%%%%%%%%%%%%%%%%%%%%%%%%%%%%%%%%%%%%%%%%%%%%%%%%%%%
\begin{frame}[fragile, allowframebreaks]
\lstset{language=Ruby, style=eclipse}
\begin{lstlisting}
#
# aws_app/ec2/ec2_setup.rb
#

require File.expand_path(File.dirname(__FILE__) + '/../config')

# get an instance of the EC2 interface using the default configuration
ec2 = sqs = AWS::EC2.new

AVAILABILITY_ZONE = 'us-east-1d'
begin
  # CentOS-6-i386-20120527-EBS-04-0c28ae8a-844c-4df2-822f-8a80374b8c8a-ami-45e9832c.1 (ami-fdbdc894)
  instance = ec2.instances.create(
      image_id: 'ami-fdbdc894',
      availability_zone: AVAILABILITY_ZONE,
      security_group_ids: 'sg-570c3f3c',
      instance_type: 't1.micro',
      block_device_mappings: [{   device_name: "/dev/sda2",
                                  ebs: {volume_size: 8, # 8 GB
                                        delete_on_termination: true
                                  }
                              }])

  begin
    puts "Waiting for instance #{instance.id} is running in availability zone #{instance.availability_zone}"
    sleep 1
  end while instance.status != :running
  puts "Launched and running instance #{instance.id} in availability zone #{instance.availability_zone}"

  # Elastic IP
  eip = ec2.elastic_ips.create
  puts "Assigned IP Adress #{eip}"
  eip.associate instance: instance.id
  puts "Associated IP Address #{eip} with instance #{instance.id}"

  # EBS
  # Create an empty 1GB volume and attach it to an instance
  volume = ec2.volumes.create(size: 1,
                              availability_zone: AVAILABILITY_ZONE)
  attachment = volume.attach_to(instance, '/dev/sdf')
  begin
  puts puts "Waiting for attachment  #{volume.id} to instance #{instance.id}"
  sleep 1
  end while attachment.status != :attaching
  puts "Attached volume #{volume.id} to instance #{instance.id}"
rescue Exception => e
  puts e.message
end
\end{lstlisting}

\end{frame}
\section{Homeworks}
\begin{frame}[fragile]
\frametitle{Homeworks}
\begin{itemize}
\item Modifica el programa \texttt{aws\_app/ec2/ec2\_setup.rb} para que a la instancia creada pueda accederse vía ssh
\item Modifica el programa \texttt{aws\_app/ec2/ec2\_setup.rb} para que libere los recursos provisionados
\end{itemize}

\end{frame}
\end{document}
