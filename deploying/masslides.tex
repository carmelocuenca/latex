%%%%%%%%%%%%%%%%%%%%%%%%%%%%%%%%%%%%%%%%%%%%%%%%%%%%%%%%%%%%%%%%%%%%%%%%%%%%%%
\subsection{Setting Up a New Rails Project - Option 1}
\begin{frame}[fragile]
\frametitle{Setting Up a New Rails Project}
To keep this guide as simple as possible, we crate a simple blog in the home directory of the user \texttt{blog\_app}

Install \acrshort{ror}

\lstset{language=shell, escapechar=!}
\begin{lstlisting}[escapechar=!]
$ gem install rails --version=4.0.0
\end{lstlisting}

\lstset{language=shell, escapechar=!}
\begin{lstlisting}[escapechar=!]
$ rails new blog_app -d postgresql
$ cd blog_app
$ rails generate scaffold post subject:string content:text
!\outputcommand{A lot of stuff}!
\end{lstlisting}

\end{frame}
%%%%%%%%%%%%%%%%%%%%%%%%%%%%%%%%%%%%%%%%%%%%%%%%%%%%%%%%%%%%%%%%%%%%%%%%%%%%%%
\subsubsection{Adapting Gemfile}
\begin{frame}[fragile]
\frametitle{Adapting Gemfile}
\lstset{language=Ruby, style=eclipse}
\begin{lstlisting}[escapechar=!]
# Gemfile
!\vdots!
gem 'unicorn', group: :production
!\vdots!
\end{lstlisting}

Then install gems via \texttt{bundler install}

\lstset{language=shell, escapechar=!}
\begin{lstlisting}[escapechar=!]
$ bundle install
\end{lstlisting}
 
To get a root URL we change to \texttt{config/routes.rb} file to this
\lstset{language=Ruby, style=eclipse}
\begin{lstlisting}
BlogApp::Application.routes.draw do
  resources :post
  root 'posts#index'
end
\end{lstlisting}
\end{frame}
%%%%%%%%%%%%%%%%%%%%%%%%%%%%%%%%%%%%%%%%%%%%%%%%%%%%%%%%%%%%%%%%%%%%%%%%%%%%%%
\subsubsection{Production Database Configuration}
\begin{frame}[fragile]
\frametitle{Production Database Configuration}
Edit the database configuration file \texttt{config/database.yml}

\lstset{language=shell, escapechar=!}
\begin{lstlisting}[escapechar=!]
development:
!\vdots!
  username: blog_app
  password: 123456
!\vdots!
production:
!\vdots!
  username: blog_app
  password: 123456
!\vdots!
\end{lstlisting}
\end{frame}
%%%%%%%%%%%%%%%%%%%%%%%%%%%%%%%%%%%%%%%%%%%%%%%%%%%%%%%%%%%%%%%%%%%%%%%%%%%%%%
\subsubsection{rake db:migrate \dots}
\begin{frame}[fragile]
\frametitle{rake db:migrate \dots}
We still need to create the databases
\lstset{language=shell, escapechar=!}
\begin{lstlisting}[escapechar=!]
$ rake db:migrate
$ rake db:setup RAILS_ENV=production
\end{lstlisting}

Start the rails server and visit the root url

\lstset{language=shell, escapechar=!}
\begin{lstlisting}[escapechar=!]
$ unicorn -p 5000 -E production
$ curl http://localhost:5000
\end{lstlisting}

Stop the rails sever

\end{frame}
%%%%%%%%%%%%%%%%%%%%%%%%%%%%%%%%%%%%%%%%%%%%%%%%%%%%%%%%%%%%%%%%%%%%%%%%%%%%%%


\subsubsection{Unicorn Configuration}
\begin{frame}[fragile, allowframebreaks]
\frametitle{Unicorn Configuration}

To be able to start Unicorn with \acrshort{rvm} environment from an \texttt{init.d} script, we now need generate a corresponding wrapper

\lstset{language=shell, escapechar=!}
\begin{lstlisting}[escapechar=!]
$ rvm wrapper 2.0.0 bootup unicorn
!\outputcommand{Regenerating 2.0.0 wrappers.}!
\end{lstlisting}

Download \href{https://github.com/defunkt/unicorn/blob/master/examples/unicorn.conf.rb}{https://github.com/defunkt/unicorn/blob/master/examples/unicorn.conf.rb}
and save as \texttt{config/unicorn.rb}. Then edit the unicorn configuration file \texttt{config/unicorn.rb}

\lstset{language=Ruby, style=eclipse}
\begin{lstlisting}[escapechar=!]
# config/unicorn.rb
# !\circled{1}!
# https://github.com/defunkt/unicorn/blob/master/examples/unicorn.conf.rb
!\vdots!
# user "unprivileged_user", "unprivileged_group"
# !\circled{2}!
user "deployer", "deployer"

# Help ensure your application will always spawn in the symlinked
# "current" directory that Capistrano sets up.
# !\circled{3}!
# working_directory "/path/to/app/current" # available in 0.94.0+
APP_PATH = "/home/deployer/blog_app"
working_directory APP_PATH # available in 0.94.0+

# listen on both a Unix domain socket and a TCP port,
# we use a shorter backlog for quicker failover when busy
# !\circled{3}!
# listen "/path/to/.unicorn.sock", :backlog => 64
listen "/tmp/.unicorn_blog_app.sock", :backlog => 64
listen 8080, :tcp_nopush => true

# nuke workers after 30 seconds instead of 60 seconds (the default)
timeout 30

# feel free to point this anywhere accessible on the filesystem
# !\circled{4}!
# pid "/path/to/app/shared/pids/unicorn.pid"
pid "/var/run/unicorn_blog_app.pid"

# By default, the Unicorn logger will write to stderr.
# Additionally, ome applications/frameworks log to stderr or stdout,
# so prevent them from going to /dev/null when daemonized here:
# !\circled{4}!
# stderr_path "/path/to/app/shared/log/unicorn.stderr.log"
# stdout_path "/path/to/app/shared/log/unicorn.stdout.log"
stderr_path APP_PATH + "/log/unicorn_blog_app.stderr.log"
stdout_path APP_PATH + "/log/unicorn_blog_app.stdout.log"
!\vdots!                                                                      
\end{lstlisting}
\end{frame}

%%%%%%%%%%%%%%%%%%%%%%%%%%%%%%%%%%%%%%%%%%%%%%%%%%%%%%%%%%%%%%%%%%%%%%%%%%%%%%
\subsubsection{Unicorn Init Script}
\begin{frame}[fragile, allowframebreaks]
\frametitle{Unicorn Init Script}

Create the init script \texttt{/etc/init.d/unicorn\_blog\_app}
\lstset{language=shell, escapechar=!}
\begin{lstlisting}[escapechar=!]
#/bin/bash

### BEGIN INIT INFO
# Provide:              unicorn
# Required-Start:       $remote_fs $syslog
# Required-Stop:        $remote_fs $syslog
# Default-Start:        2 3 4 5
# Default-Stop:         0 1 6
# Short-Description:    Unicorn webserver
# Description:          Unicorn webserver for blog_app
### END INIT INFO

UNICORN=/home/deployer/.rvm/bin/bootup_unicorn
UNICORN_ARGS="-D -c /home/deployer/blog_app/config/unicorn.rb -p 5000 -E production"
KILL=/bin/kill
PID=/var/run/unicorn_blog_app.pid

sig() {
  test -s "$PID" && kill -$1 `cat $PID`
}

case "$1" in
  start)
    echo "Starting unicorn..."
    $UNICORN $UNICORN_ARGS
    ;;
  stop)
    echo "Stoping unicorn..."
    sig QUIT & exit 0
    echo >&2 "Not running"
    ;;
  restart)
    $0 stop
    $0 start
    ;;
  status)
    ;;
  *)
    echo "Usage: $0 {start|stop|restart|status}"
esac             
\end{lstlisting}
\end{frame}


%%%%%%%%%%%%%%%%%%%%%%%%%%%%%%%%%%%%%%%%%%%%%%%%%%%%%%%%%%%%%%%%%%%%%%%%%%%%%%
\subsubsection{\texttt{/etc/hosts}}
\begin{frame}[fragile]
\frametitle{\texttt{/etc/hosts}}
Edit \texttt{/etc/hosts} with the following content in order to resolve the server \texttt{blog\_app.com}
\lstset{language=shell, escapechar=!}
\begin{lstlisting}[escapechar=!]
127.0.0.1   localhost localhost.localdomain localhost4 localhost4.localdomain4
::1         localhost localhost.localdomain localhost6 localhost6.localdomain6
127.0.0.1   blog_app.com # !\circled{1}!
\end{lstlisting}
\end{frame}


%%%%%%%%%%%%%%%%%%%%%%%%%%%%%%%%%%%%%%%%%%%%%%%%%%%%%%%%%%%%%%%%%%%%%%%%%%%%%%
\subsubsection{Apache Configuration}
\begin{frame}[fragile, allowframebreaks]
\frametitle{Apache Configuration}

Add a Virtual Host in Apache configuration file \texttt{/etc/httpd/conf/httpd.conf}

\lstset{language=shell, escapechar=&}
\begin{lstlisting}
&\vdots&
&\circled{1}&
NameVirtualHost *:80
&\vdots&
&\circled{2}&
<VirtualHost *:80>
  ServerAdmin webmaster@blog_app.com
  DocumentRoot /home/deployer/blog_app
  ServerName blog_app.com
  ErrorLog logs/blog_app.example.com-error_log
  CustomLog logs/blog_app.example.com-access_log common

  RewriteEngine On

# Redirect all non-static requests to unicorn
  RewriteCond %{DOCUMENT_ROOT}/%{REQUEST_FILENAME} !-f 
  RewriteRule ^/(.*)$ http://127.0.0.1:5000%{REQUEST_URI} [P,QSA,L]
</VirtualHost>
\end{lstlisting}


Restart Apache daemon

\lstset{language=shell, escapechar=!}
\begin{lstlisting}[escapechar=!]
# /etc/init.d/httpd restart
\end{lstlisting}

\end{frame}




%%%%%%%%%%%%%%%%%%%%%%%%%%%%%%%%%%%%%%%%%%%%%%%%%%%%%%%%%%%%%%%%%%%%%%%%%%%%%%