\documentclass{beamer}
\usepackage[british,spanish]{babel}
\usepackage[utf8]{inputenc}

\usepackage{hyperref}
%\hypersetup{colorlinks=false,linkbordercolor=red,linkcolor=green,pdfborderstyle={/S/U/W 1}}

\usepackage{multirow}

\usepackage{textcomp}

\usepackage{listings}
\lstloadlanguages{Ruby}

\usepackage{adjustbox}
\usepackage{lstcustom}

\usepackage{amsmath}

\usepackage{color}
\definecolor{light-gray}{gray}{0.80}
\definecolor{lstbackgroundshellcolor}{named}{light-gray}

\usepackage{tikz}
\newcommand*\circled[1]{\tikz[baseline=(char.base)]{
            \node[shape=circle,draw,inner sep=2pt] (char) {#1};}}

\usepackage[normalem]{ulem}

%\usepackage[acronym,xindy,toc]{glossaries}

\usepackage[acronym,xindy,toc]{glossaries}
\makeglossaries
%\usepackage[xindy]{imakeidx}
%\makeindex


\newcommand{\comment}[2]{#2}

\newcommand{\commandinline}[1]{\lstinline[basicstyle=\small\lstfontfamily]{#1}}
\newcommand{\outputcommand}[1]{\color{darkgreen}{#1}}

\graphicspath{ {./images/} }

\title{Deploying a Ruby on Rails Application in an Amazon EC2 Virtual Machine}
%\subtitle[short subtitle]{long subtitle}
\author[C. Cuenca, F. Quintana]{Carmelo Cuenca-Hernández and Francisca Quintana-Domínguez}
%\institute{Escuela Universitaria de Informática}
%\date[04/2013]{Abril - 2013}
\date{}
\titlegraphic{\includegraphics[width=0.5 \textwidth]{./images/logo_ulpgc_version_horizontal_rgb.eps}}



\pgfdeclareimage[width=2.0\baselineskip]{ulpgc-logo}{./images/logosimbolo_secundario_version_vertical}
\setbeamertemplate{footline}{\raisebox{-2ex}{\pgfuseimage{ulpgc-logo}}
  \usebeamerfont{date in head/foot}\insertshortdate{}\hfill
  \usebeamertemplate{navigation symbols}\hfill
  \insertframenumber{}/\inserttotalframenumber}
\setbeamertemplate{sidebar right}{}


\usetheme{Antibes}
%\usetheme{Berlin}

%\usetheme{Warsaw}
%\usecolortheme{albatross}

\selectlanguage{british}

\begin{document}


\begin{frame}
	\titlepage
\end{frame}


\section*{Outline}
\begin{frame}
  \frametitle{Outline}
  %\tableofcontents%[part=1,pausesections]
  \tableofcontents[currentsection,currentsubsection, sectionstyle=show] 
  %\tableofcontents[currentsection,sectionstyle=show,hideothersubsections]
\end{frame}

%%%%%%%%%%%%%%%%%%%%%%%%%%%%%%%%%%%%%%%%%%%%%%%%%%%%%%%%%%%%%%%%%%%%%%%%%%%%%%
%\newacronym{<label>}{<abbrv>}{<full>}
%\glsreset{<label>}
%\glsresetall
%\acrlong{<label>}
%\acrfull{<label>}
%\acrshort{<label>}
%\input{../glossary}

\newacronym{aws}{AWS}{Amazon Web Services}
\newacronym{css}{CSS}{cascading style sheets}
\newacronym{ebs}{EBS}{Elastic Block Storage}
\newacronym{ec2}{EC2}{Amazon Elastic Compute Cloud}
\newacronym{elb}{ELB}{Elastic Load Balancing}
\newacronym{ror}{RoR}{{\href{http://rubyonrails.org/}{Ruby on Rails}}}
\newacronym{rds}{RDS}{Relational Database Service}
\newacronym{rvm}{RVM}{{\href{https://rvm.io/}{Ruby Version Manager}}}
\newacronym{s3}{S3}{Simple Storage Service}
\newacronym{sqs}{SQS}{Amazon Simple Queue Service}
%%%%%%%%%%%%%%%%%%%%%%%%%%%%%%%%%%%%%%%%%%%%%%%%%%%%%%%%%%%%%%%%%%%%%%%%%%%%%%
%%%%%%%%%%%%%%%%%%%%%%%%%%%%%%%%%%%%%%%%%%%%%%%%%%%%%%%%%%%%%%%%%%%%%%%%%%%%%%
%\section*{Web Server in Production Mode}

%%%%%%%%%%%%%%%%%%%%%%%%%%%%%%%%%%%%%%%%%%%%%%%%%%%%%%%%%%%%%%%%%%%%%%%%%%%%%%

\section{Setting Up a Rails Project}
\begin{frame}[fragile, allowframebreaks]
\frametitle{Setting Up a Rails Project}
\begin{itemize}
\item To keep this presentation as simple as possible, we clone an example applicacion in the \texttt{public\_html} directory 

\lstset{language=shell, escapechar=!}
\begin{lstlisting}[escapechar=!]
$ cd ~/public_html
$ rvm ruby-2.0.0-p353@railstutorial_rails_4_0 --create
$ gem install rails --version=4.0.0
$ rails new depot_app -d postgresql
\end{lstlisting}

\lstset{language=shell, escapechar=!}
\begin{lstlisting}[escapechar=!]
$ cd ~/depot_app
$ rails generate scaffold company name:string
$ rails generate scaffold employee company_id:integer last_name:string first_name:string phone_number:string
\end{lstlisting}

# This file should contain all the record creation needed to seed the database with its default values.
# The data can then be loaded with the rake db:seed (or created alongside the db with db:setup).
#
# Examples:
#
#   cities = City.create([{ name: 'Chicago' }, { name: 'Copenhagen' }])
#   Mayor.create(name: 'Emanuel', city: cities.first)

30.times do
  company = Company.new name: Faker::Company.name
  if company.save
    SecureRandom::random_number(100).times do
      company.employees.create first_name: Faker::Name.first_name,
        last_name: Faker::Name.last_name,
        phone_number: Faker::PhoneNumber.phone_number
    end
  end
end

la linea en Gemfile gem 'faker'

bundle install

rake db:create
rake db:migrate

Editar los ficheros de los modelos

class Company < ActiveRecord::Base
  has_many :employees, dependent: :destroy

  validates :name,
    presence: true,
    uniqueness: true

  def to_s
    name
  end

end

class Employee < ActiveRecord::Base

  belongs_to :company, touch: true

  validates :first_name,
    presence: true
  validates :last_name,
    presence: true
  validates :company,
    presence: true

  def to_s
    "#{first_name} #{last_name}"
  end

end

rails generate scaffold employee last_name:string first_name:string phone_number:string
<p id="notice"><%= notice %></p>

<p>
  <strong>Name:</strong>
  <%= @company.name %>
</p>

<%= link_to 'Edit', edit_company_path(@company) %> |
<%= link_to 'Back', companies_path %>

<h1>Listing employees</h1>

<table>
  <thead>
    <tr>
      <th>Last Name</th>
      <th>Fist Name</th>
      <th>Phone Number</th>
    </tr>
  </thead>

  <tbody>
    <% @company.employees.each do |employee|%>
      <tr>
        <td><%= employee.last_name %></td>
        <td><%= employee.first_name %></td>
        <td><%= employee.phone_number %></td>
      </tr>
    <% end%>
  </tbody>
</table>




\end{frame}
%%%%%%%%%%%%%%%%%%%%%%%%%%%%%%%%%%%%%%%%%%%%%%%%%%%%%%%%%%%%%%%%%%%%%%%%%%%%%%

\end{document}